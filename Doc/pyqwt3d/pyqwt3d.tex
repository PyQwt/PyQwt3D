\documentclass{manual}

% Links
\newcommand{\QwtPlotDdd}{\ulink{QwtPlot3D}
  {http://qwtplot3d.sourceforge.net}}
\newcommand{\mailinglist}{\ulink{mailing list}
  {mailto:pyqwt-users@lists.sourceforge.net}}

% Source code
\newcommand{\snapshot}{\ulink{snapshot}
  {http://www.river-bank.demon.co.uk/download/snapshots}}
\newcommand{\Numeric}{\ulink{Numeric}
  {http://www.numpy.org}}
\newcommand{\NumericTarGz}{\ulink{Numeric-23.6.tar.gz}
  {http://prdownloads.sourceforge.net/numpy/Numeric-23.6.tar.gz?download}}
\newcommand{\numarray}{\ulink{numarray}
  {http://www.stsci.edu/resources/software_hardware/numarray}}
\newcommand{\numarrayTarGz}{\ulink{numarray-1.1.1.tar.gz}
  {http://prdownloads.sourceforge.net/numpy/numarray-1.1.1.tar.gz?download}}
\newcommand{\optik}{\ulink{optik}
  {http://optik.sourceforge.net}}
\newcommand{\NewSip}{\ulink{sip-4.1.1.tar.gz}
  {http://pyqwt.sourceforge.net/support/sip-4.1.1.tar.gz}}
\newcommand{\PyQtGpl}{\ulink{PyQt-x11-gpl-3.13.tar.gz}
  {http://pyqwt.sourceforge.net/support/PyQt-x11-gpl-3.13.tar.gz}}
\newcommand{\PyQtMac}{\ulink{PyQt-mac-gpl-3.13.tar.gz}
  {http://pyqwt.sourceforge.net/support/PyQt-mac-gpl-3.13.tar.gz}}
\newcommand{\PyQtCom}{\ulink{PyQt-commercial}
  {http://www.riverbankcomputing.co.uk/pyqt/buy.php}}
\newcommand{\PyQwtDddTarGz}{\ulink{PyQwt3D-0.1.tar.gz}
  {http://prdownloads.sourceforge.net/pyqwt/PyQwt3D-0.1.tar.gz?download}}
\newcommand{\QwtPlotDddTgz}{\ulink{qwtplot3d-0.2.4-beta.tgz}
  {http://prdownloads.sourceforge.net/qwtplot3d/qwtplot3d-0.2.4-beta.tgz?download}}
\newcommand{\QwtPlotDddZip}{\ulink{qwtplot3d-0.2.4-beta.zip}
  {http://prdownloads.sourceforge.net/qwtplot3d/qwtplot3d-0.2.4-beta.zip?download}}

\newcommand{\PrerequisitesEnd}{
  To exploit the full power of the PyQwt3D, you should install at
  least one the Numerical Python extensions:
  \Numeric{} or its successor \numarray{}.
  I am using \NumericTarGz{} and \numarrayTarGz{}.
  Versions of Numeric later than 21.0 are supported.
  Numarray is newer and therefore less stable than Numeric, so get the latest!
  \begin{notice}[warning]
    PyQwt3D may not work with numarray on Linux systems, possibly due to a bug
    in the floating point excepion handling of glibc (occurs at least on
    Mandrake-10.0 and SuSE-9.0).
    More information is to be found
    \ulink{here}{http://sourceforge.net/mailarchive/message.php?msg_id=9914816}
    and in related posts.
    Your mileage may vary: PyQwt3D works with numarray on SuSE-9.1.
  \end{notice}
}

\newcommand{\Future}{
  \begin{notice}[warning]
    The documentation is for the future PyQwt3D-0.1 which is only available
    from CVS.
  \end{notice}
}

%\renewcommand{\Future}{}

\title{PyQwt3D Manual}

% boilerplate.tex?
\author{Gerard Vermeulen}

\date{\today}
\release{0.1}
\setshortversion{0.1}

\makeindex

\begin{document}

\maketitle

% This makes the contents more accessible from the front page of the HTML.
\ifhtml
\chapter*{Front Matter \label{front}}
\fi

Copyright \copyright{} 2004 Gerard Vermeulen

PyQwt3D is free software; you can redistribute it and/or modify it under the
terms of the GNU General Public License as published by the Free Software
Foundation; either version 2 of the License, or (at your option) any later
version.

PyQwt3D is distributed in the hope that it will be useful, but WITHOUT ANY
WARRANTY; without even the implied warranty of MERCHANTABILITY or FITNESS
FOR A PARTICULAR PURPOSE.  See the GNU  General Public License for more
details.

You should have received a copy of the GNU General Public License along with
PyQwt3D; if not, write to the Free Software Foundation, Inc., 59 Temple Place,
Suite 330, Boston, MA 02111-1307, USA.

In addition, as a special exception, Gerard Vermeulen gives permission to
link PyQwt3D dynamically with commercial, non-commercial or educational 
versions of Qt, PyQt and sip, and distribute PyQwt3D in this form, provided
that equally powerful versions of Qt, PyQt and sip have been released under
the terms of the GNU General Public License.

If PyQwt3D is dynamically linked with commercial, non-commercial or
educational versions of Qt, PyQt and sip, PyQwt3D becomes a free plug-in
for a non-free program.



\begin{abstract}

\noindent
PyQwt3D is a set of Python bindings for the \QwtPlotDdd{} library.

\end{abstract}

\tableofcontents

\chapter{Introduction\label{introduction}}

PyQwt3D is a set of Python bindings for the \QwtPlotDdd{} library.

\chapter{Installation\label{installation}}

\section{Installation prerequisites\label{prerequisites}}

\Future{}

Build prerequisites for \PyQwtDddTarGz{} are:
\begin{enumerate}
\item
  \ulink{Python}{http://www.python.org}.\\
  Supported versions: Python-2.4.x and Python-2.3.x.\\
  PyQwt3D will also work with earlier versions of Python, if you install
  \optik{}-1.4.1 or later.
\item
  \QwtPlotDdd{}.\\
  Supported versions: QtPlot3D-0.2.4.
\item
  \ulink{Qt}{http://www.trolltech.com}.\\
  Supported versions: Qt-3.3.x, Qt-3.2.x, Qt-3.1.x, Qt-3.0.x.
\item
  \NewSip{}.\\
  Supported versions: SIP-4.1.1, -4.1, -4.0.1, and -4.0,
  but PyQwt3D built with SIP-4.1.x is more powerful than PyQwt3D built with
  SIP-4.0.x.\\
  You may also try a \snapshot{} at your own risk.
\item
  \PyQtGpl{}, \PyQtMac{} or \PyQtCom{}.\\
  Supported versions: PyQt-3.13, -3.12, -3.11, and -3.10,
  but the most recent version gets most testing.\\
  You may also try a \snapshot{} at your own risk.
\end{enumerate}

\PrerequisitesEnd{}
  

\section{Installation\label{install}}

\subsection{Install on Windows with MSVC\label{win-install}}

\Future{}

\begin{notice}[warning]
  PyQwt3D only be tested on Windows with MSCV-6.0 and assumes that MSVC-7.x
  requires the same workarounds. I rely on your feedback to fix eventual
  installation problems, since I have no commercial license for Qt and PyQt.
\end{notice}

The installation procedure contains three steps:
\begin{enumerate}
\item
  Unpack \PyQwtDddTarGz{} and \QwtPlotDddZip{}.
\item
  Do a quick start to test the installation by running the commands:
\begin{verbatim}
cd PyQwt3D-0.1
cd configure
python configure.py -Q C:\sources\of\qwtplot3
nmake
nmake install
\end{verbatim}
  where the folder
  \file{C:\textbackslash{}sources\textbackslash{}of\textbackslash{}qwtplot3d}
  must contain the folders \file{3rdparty}, \file{include} and \file{src}.
  You can also edit the \file{go.bat} to suit your setup.
\item
  Fine tune (optional) by running the commands:
\begin{verbatim}
python configure.py -Q C:\sources\of\qwtplot3d -l zlib -D GL2PS_HAVE_ZLIB
nmake
nmake install
\end{verbatim}
    to enable compression of PostScript and PDF files. Add
\begin{verbatim}
-L C:\folder\containing\zlib.lib
\end{verbatim}
    to the \file{configure.py} options, if the linker fails to find the zlib
    library.
\end{enumerate}

Contact the \mailinglist{}, if you run into problems.

\begin{notice}[note]
  The configure.py script takes many options. The command
\begin{verbatim}
python configure.py -h
\end{verbatim}
  displays a full list of the available options:
  \verbatiminput{configure.help}
\end{notice}


\subsection{Installation on \POSIX{} and MacOS/X\label{posix-install}}

\Future{}

The installation procedure contains three steps:
\begin{enumerate}
\item
  Unpack \PyQwtDddTarGz{} and \QwtPlotDddTgz{}.
\item
  Do a quick start to test the installation by running the commands:
\begin{verbatim}
cd PyQwt3D-0.1
cd configure
python configure.py -Q /sources/of/qwtplot3
make
make install
\end{verbatim}
  where the directory
  \file{/sources/of/qwtplot3d} must contain the directories \file{3rdparty},
  \file{include} and \file{src}.\\
  The installation will fail if Qt has been configured without runtime type
  information (RTTI).  In this case, run the commands:
\begin{verbatim}
cd PyQwt3D-0.1
cd configure
python configure.py -Q /sources/of/qwtplot3 --extra-cxxflags=-frtti
make
make install
\end{verbatim}
  where \code{-frtti} enables RTTI for g++.  Check your compiler documention
  for other C++ compilers.
\item
  Fine tune (optional)
  \begin{itemize}
    \item
      to enable compression of PostScript and PDF files by running the
      commands:
\begin{verbatim}
python configure.py -Q /sources/of/qwtplot3d -l z -D GL2PS_HAVE_ZLIB
make
make install
\end{verbatim}
      Add
\begin{verbatim}
-L /directory/with/libz.*
\end{verbatim}
      to the \file{configure.py} options, if the linker fails to find the zlib
      library.
    \item
      to use a the QwtPlot3D library on your system by running the commands:
\begin{verbatim}
rm -rf Qwt3D
python configure.py -I /usr/include/qwtplot3d
make
make install
\end{verbatim}
    Add
\begin{verbatim}
-L /directory/with/libqwtplot3d.*
\end{verbatim}
    to the \file{configure.py} options, if the linker fails to find the
    QwtPlot3D library.
  \end{itemize}
\end{enumerate}

Contact the \mailinglist{}, if you run into problems.

\begin{notice}[note]
  The configure.py script takes many options. The command
\begin{verbatim}
python configure.py -h
\end{verbatim}
  displays a full list of the available options:
  \verbatiminput{configure.help}
\end{notice}


\chapter{PyQwt3D Module Reference \label{reference}}

%begin{latexonly}
\renewcommand{\indexname}{Index}
%end{latexonly}
\input{\jobname.ind}

\end{document}

%% Local Variables:
%% fill-column: 79
%% End:
