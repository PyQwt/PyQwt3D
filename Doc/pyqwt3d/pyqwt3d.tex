\documentclass{manual}

% Links
\newcommand{\QwtPlotDDD}{\ulink{QwtPlot3D}{http://qwtplot3d.sourceforge.net}}

% Source code
\newcommand{\snapshot}{\ulink{snapshot}
  {http://www.river-bank.demon.co.uk/download/snapshots}}
\newcommand{\Numeric}{\ulink{Numeric}
  {http://www.numpy.org}}
\newcommand{\NumericTarGz}{\ulink{Numeric-23.6.tar.gz}
  {http://prdownloads.sourceforge.net/numpy/Numeric-23.6.tar.gz?download}}
\newcommand{\numarray}{\ulink{numarray}
  {http://www.stsci.edu/resources/software_hardware/numarray}}
\newcommand{\numarrayTarGz}{\ulink{numarray-1.1.1.tar.gz}
  {http://prdownloads.sourceforge.net/numpy/numarray-1.1.1.tar.gz?download}}
\newcommand{\optik}{\ulink{optik}
  {http://optik.sourceforge.net}}
\newcommand{\NewSip}{\ulink{sip-4.1.1.tar.gz}
  {http://pyqwt.sourceforge.net/support/sip-4.1.1.tar.gz}}
\newcommand{\PyQtGpl}{\ulink{PyQt-x11-gpl-3.13.tar.gz}
  {http://pyqwt.sourceforge.net/support/PyQt-x11-gpl-3.13.tar.gz}}
\newcommand{\PyQtMac}{\ulink{PyQt-mac-gpl-3.13.tar.gz}
  {http://pyqwt.sourceforge.net/support/PyQt-mac-gpl-3.13.tar.gz}}
\newcommand{\PyQtCom}{\ulink{PyQt-commercial}
  {http://www.riverbankcomputing.co.uk/pyqt/buy.php}}
\newcommand{\PyQwtDDDTarGz}{\ulink{PyQwt3D-0.1.tar.gz}
  {http://prdownloads.sourceforge.net/pyqwt/PyQwt3D-0.1.tar.gz?download}}

\newcommand{\PrerequisitesEnd}{
  To exploit the full power of the PyQwt3D, you should install at
  least one the Numerical Python extensions:
  \Numeric{} or its successor \numarray{}.
  I am using \NumericTarGz{} and \numarrayTarGz{}.
  Versions of Numeric later than 21.0 are supported.
  Numarray is newer and therefore less stable than Numeric, so get the latest!
  \begin{notice}[warning]
    PyQwt3D may not work with numarray on Linux systems, possibly due to a bug
    in the floating point excepion handling of glibc (occurs at least on
    Mandrake-10.0 and SuSE-9.0).
    More information is to be found
    \ulink{here}{http://sourceforge.net/mailarchive/message.php?msg_id=9914816}
    and in related posts.
    Your mileage may vary: PyQwt3D works with numarray on SuSE-9.1.
  \end{notice}
}

\newcommand{\Future}{
  \begin{notice}[warning]
    The documentation is for the future PyQwt3D-0.1 which is only available
    from CVS.
  \end{notice}
}

%\renewcommand{\Future}{}

\title{PyQwt3D Manual}

% boilerplate.tex?D
\author{Gerard Vermeulen}

\date{\today}
\release{0.1}
\setshortversion{0.1}

\makeindex

\begin{document}

\maketitle

% This makes the contents more accessible from the front page of the HTML.
\ifhtml
\chapter*{Front Matter \label{front}}
\fi

\input{copyright}


\begin{abstract}

\noindent
PyQwt3D is a set of Python bindings for the \QwtPlotDDD{} library.

\end{abstract}

\tableofcontents

\chapter{Introduction\label{introduction}}

PyQwt3D is a set of Python bindings for the \QwtPlotDDD{} library.

\chapter{Installation\label{installation}}

\section{Build prerequisites\label{prerequisites}}

\Future{}

Build prerequisites for \PyQwtDDDTarGz{} are:
\begin{enumerate}
\item
  \ulink{Python}{http://www.python.org}.\\
  Supported versions: Python-2.4.x and Python-2.3.x.\\
  PyQwt3D will also work with earlier versions of Python, if you install
  \optik{}-1.4.1 or later.
\item
  \QwtPlotDDD{}.\\
  Supported versions: QtPlot3D-0.2.4.
\item
  \ulink{Qt}{http://www.trolltech.com}.\\
  Supported versions: Qt-3.3.x, Qt-3.2.x, Qt-3.1.x, Qt-3.0.x.
\item
  \NewSip{}.\\
  Supported versions: SIP-4.1.1, -4.1, -4.0.1, and -4.0,
  but PyQwt3D built with SIP-4.1.x is more powerful than PyQwt3D built with
  SIP-4.0.x.\\
  You may also try a \snapshot{} at your own risk.
\item
  \PyQtGpl{}, \PyQtMac{} or \PyQtCom{}.\\
  Supported versions: PyQt-3.13, -3.12, -3.11, and -3.10,
  but the most recent version gets most testing.\\
  You may also try a \snapshot{} at your own risk.
\end{enumerate}

\PrerequisitesEnd{}
  

\section{Build and install\label{build}}

\Future{}

\begin{enumerate}
\item
  Unpack \PyQwtDDDTarGz{}
\item
  Configure PyQwt3D by running the following commands:
\begin{verbatim}
cd PyQwt3D-0.1
cd configure
python configure.py [options]
\end{verbatim}
  This assumes that the correct Python interpreter is on your path.
  The configure.py script takes many options.
  The command
\begin{verbatim}
python configure.py -h
\end{verbatim}
displays a full list of the available options:
\verbatiminput{configure.help}
  \begin{notice}[note]
    PyQwt3D requires compilation of QwtPlot3D and PyQwt3D with runtime type
    information (RTTI).
  \end{notice}
  \begin{notice}[note]
    PyQwt3D in combination MSVC and Windows requires:
\begin{verbatim}
python configure.py -Q C:\sources\of\qwtplot3d
\end{verbatim}
    The folder
    \file{C:\textbackslash{}sources\textbackslash{}of\textbackslash{}qwtplot3d}
    must contain the folders \file{3rdparty}, \file{include} and \file{src}.
    Edit \file{configure/go.bat} to suit your setup.
    Use:
\begin{verbatim}
python configure.py -Q C:\sources\of\qwtplot3d \
    --extra-libs=zlib --extra-defines=GL2PS_HAVE_ZLIB
\end{verbatim}
    to enable compression of PostScript and PDF files.
\end{notice}

\item
  The next step is to build PyQwt3D using your platform's make command.
\begin{verbatim}
make
\end{verbatim}
  or
\begin{verbatim}
nmake
\end{verbatim}
  for MSVC.
\item
  The final step is to install PyQwt3D by running the following command.
\begin{verbatim}
make install
\end{verbatim}
  or
\begin{verbatim}
nmake install
\end{verbatim}
  for MSVC.
\end{enumerate}

\chapter{PyQwt3D Module Reference \label{reference}}

%begin{latexonly}
\renewcommand{\indexname}{Index}
%end{latexonly}
\input{\jobname.ind}

\end{document}

%% Local Variables:
%% fill-column: 79
%% End:
