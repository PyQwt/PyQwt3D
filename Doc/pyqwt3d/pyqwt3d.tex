\documentclass{manual}

% Links
\newcommand{\QwtPlotDdd}{\ulink{QwtPlot3D}
  {http://qwtplot3d.sourceforge.net}}
\newcommand{\QwtPlotDddApi}{\ulink{QwtPlot3D API documentation}
  {http://qwtplot3d.sourceforge.net/web/navigation/api_frame.html}}
\newcommand{\QwtPlotDddManual}{\ulink{QwtPlot3D manual}
  {http://qwtplot3d.sourceforge.net/web/navigation/manual_frame.html}}
\newcommand{\ZLib}{\ulink{ZLib}
  {http://www.gzip.org/zlib}}
\newcommand{\mailinglist}{\ulink{mailing list}
  {mailto:pyqwt-users@lists.sourceforge.net}}

% Source code
\newcommand{\NumPy}{\ulink{NumPy}
  {http://www.numpy.org}}
\newcommand{\NumPyTarGz}{\ulink{numpy-1.0.3}
  {http://prdownloads.sourceforge.net/numpy/numpy-1.0.3.tar.gz?download}}
\newcommand{\numarray}{\ulink{numarray}
  {http://www.stsci.edu/resources/software_hardware/numarray}}
\newcommand{\numarrayTarGz}{\ulink{numarray-1.5.2.tar.gz}
  {http://prdownloads.sourceforge.net/numpy/numarray-1.5.2.tar.gz?download}}
\newcommand{\Numeric}{\ulink{Numeric}
  {http://www.numpy.org}}
\newcommand{\NumericTarGz}{\ulink{Numeric-24.2.tar.gz}
  {http://prdownloads.sourceforge.net/numpy/Numeric-24.2.tar.gz?download}}
\newcommand{\NewSip}{\ulink{sip-4.7.tar.gz}
  {http://pyqwt.sourceforge.net/support/sip-4.7.tar.gz}}
\newcommand{\PyQtGpl}{\ulink{PyQt-x11-gpl-3.17.3.tar.gz}
  {http://pyqwt.sourceforge.net/support/PyQt-x11-gpl-3.17.3.tar.gz}}
\newcommand{\PyQtMac}{\ulink{PyQt-mac-gpl-3.17.2.tar.gz}
  {http://pyqwt.sourceforge.net/support/PyQt-mac-gpl-3.17.3/tar.gz}}
\newcommand{\PyQtFGpl}{\ulink{PyQt4-x11-gpl-4.3.tar.gz}
  {http://pyqwt.sourceforge.net/support/PyQt-x11-gpl-4.3.tar.gz}}
\newcommand{\PyQtFMac}{\ulink{PyQt4-mac-gpl-4.3.tar.gz}
  {http://pyqwt.sourceforge.net/support/PyQt-mac-gpl-4.3.tar.gz}}
\newcommand{\PyQtCom}{\ulink{PyQt-commercial}
  {http://www.riverbankcomputing.co.uk/pyqt/buy.php}}
\newcommand{\PyQwtDddTarGz}{\ulink{PyQwt3D-0.1.6.tar.gz}
  {http://prdownloads.sourceforge.net/pyqwt/PyQwt3D-0.1.6.tar.gz?download}}
\newcommand{\QwtPlotDddTgz}{\ulink{qwtplot3d-0.2.7.tgz}
  {http://prdownloads.sourceforge.net/qwtplot3d/qwtplot3d-0.2.7.tgz?download}}
\newcommand{\QwtPlotDddZip}{\ulink{qwtplot3d-0.2.7.zip}
  {http://prdownloads.sourceforge.net/qwtplot3d/qwtplot3d-0.2.7.zip?download}}
\newcommand{\Examples}{\ulink{PyQwt3D examples}
  {http://pyqwt.sourceforge.net/pyqwt3d-examples.html}}
\newcommand{\MacPorts}{\ulink{MacPorts}
  {http://www.macports.org}}

% Installers for MS-Windows
\newcommand{\PythonExe}{\ulink{python-2.5.1.msi}
  {http://www.python.org/ftp/python/2.5/python-2.5.1.msi}}
\newcommand{\NumPyExe}{\ulink{numpy-1.0.3.win32-py2.5.exe}
  {http://prdownloads.sourceforge.net/numpy/numpy-1.0.3.win32-py2.5.exe}}
\newcommand{\QtOldExe}{\ulink{qt-win-opensource-4.2.2-mingw.exe}
  {ftp://ftp.trolltech.com/qt/source/qt-win-opensource-4.2.2-mingw.exe}}
\newcommand{\QtNewExe}{\ulink{qt-win-opensource-4.2.3-mingw.exe}
  {ftp://ftp.trolltech.com/qt/source/qt-win-opensource-4.2.3-mingw.exe}}
\newcommand{\DevCpp}{\ulink{Dev-C++}
  {http://prdownloads.sourceforge.net/dev-cpp/devcpp-4.9.9.2_setup.exe}}
\newcommand{\PyQtOldExe}{\ulink{PyQt-gpl-4.1.1-Py2.5-Qt4.2.2.exe}
  {http://pyqwt.sourceforge.net/support/PyQt-gpl-4.1.1-Py2.5-Qt4.2.2.exe}}
\newcommand{\PyQtNewExe}{\ulink{PyQt-gpl-4.2-Py2.5-Qt4.2.3.exe}
  {http://pyqwt.sourceforge.net/support/PyQt-gpl-4.2-Py2.5-Qt4.2.3.exe}}
\newcommand{\PyQwtNewExe}{\ulink{PyQwt3D-0.1.6-Python2.5-Qt4.2.3-PyQt-4.2-NumPy1.0.3.exe}
  {http://prdownloads.sourceforge.net/pyqwt/PyQwt3D-0.1.6-Python2.5-Qt4.2.3-PyQt-4.2-NumPy1.0.3.exe}}



\newcommand{\PrerequisitesEnd}{
  To exploit the full power of the PyQwt3D, you should install at
  least one the Numerical Python extensions:
  \NumPy{}, \numarray{}, \Numeric{}.

  I am testing PyQwt3D with \NumPyTarGz{}, \numarrayTarGz{}, and
  \NumericTarGz{}. \NumPy{} is actively developed and recommended;
  \numarray{} and \Numeric{} are unmaintained.

  \begin{notice}[note]
    PyQwt3D contains a patched version of QwtPlot3D-0.2.7 to enable the
    SVG and PGF vector output file formats and the patched version of
    QwtPlot3D-0.2.7 is required when you want to use the SVG and PGF formats. 
    The PyQwt3D extension module containing statically linked sources of
    QwtPlot3D-0.2.7 coexists very well with system wide shared libraries
    of any version of QwtPlot3D.
  \end{notice}
}

\newcommand{\Future}{
  \begin{notice}[warning]
    The documentation is for the future PyQwt3D-0.1.6 which has not yet been
    released.  Please refer, to the documentation in the release that you are
    using.
  \end{notice}
}

\renewcommand{\Future}{}


\title{PyQwt3D Manual}

% boilerplate.tex?
\author{Gerard Vermeulen}

\date{\today}
\release{0.1.6}
\setshortversion{0.1.6}

\makeindex

\begin{document}

\maketitle

% This makes the contents more accessible from the front page of the HTML.
\ifhtml
\chapter*{Front Matter \label{front}}
\fi

Copyright \copyright{} 2004 Gerard Vermeulen

PyQwt3D is free software; you can redistribute it and/or modify it under the
terms of the GNU General Public License as published by the Free Software
Foundation; either version 2 of the License, or (at your option) any later
version.

PyQwt3D is distributed in the hope that it will be useful, but WITHOUT ANY
WARRANTY; without even the implied warranty of MERCHANTABILITY or FITNESS
FOR A PARTICULAR PURPOSE.  See the GNU  General Public License for more
details.

You should have received a copy of the GNU General Public License along with
PyQwt3D; if not, write to the Free Software Foundation, Inc., 59 Temple Place,
Suite 330, Boston, MA 02111-1307, USA.

In addition, as a special exception, Gerard Vermeulen gives permission to
link PyQwt3D dynamically with commercial, non-commercial or educational 
versions of Qt, PyQt and sip, and distribute PyQwt3D in this form, provided
that equally powerful versions of Qt, PyQt and sip have been released under
the terms of the GNU General Public License.

If PyQwt3D is dynamically linked with commercial, non-commercial or
educational versions of Qt, PyQt and sip, PyQwt3D becomes a free plug-in
for a non-free program.



\begin{abstract}

\noindent
PyQwt3D is a set of Python bindings for the \QwtPlotDdd{} library which extends
the Qt framework with widgets to visualize 3-dimensional data.
It allows you to integrate PyQt, Qt, QwtPlot3D, the Numerical Python
extensions, and optionally SciPy in a GUI Python application or in an
interactive Python session.

\end{abstract}

\tableofcontents

\chapter{Introduction\label{introduction}}

\Future{}

PyQwt3D is a set of Python bindings for the \QwtPlotDdd{} library which extends
the Qt framework with widgets to visualize 3-dimensional data.

It wraps also a small subset of the OpenGL API.

The \Examples{} show how to program with PyQwt3D.


\chapter{Installation\label{installation}}

\Future{}

\begin{notice}[note]
  PyQwt3D for Qt-3 can coexist with PyQwt3D for Qt-4. The statement
  \begin{verbatim}
import Qwt3D
  \end{verbatim}
  imports PyQwt3D for Qt-3 and the statement
  \begin{verbatim}
import PyQt4.Qwt3D
  \end{verbatim}
  imports PyQwt3D for Qt-4 
\end{notice}


\section{Installation prerequisites\label{prerequisites}}

\Future{}

Installation prerequisites for \PyQwtDddTarGz{} are:
\begin{enumerate}
\item
  \ulink{Python}{http://www.python.org}.\\
  Supported versions: Python-2.5.x, and Python-2.4.x.
\item
  \ulink{Qt}{http://www.trolltech.com}.\\
  Supported versions: Qt-4.3.x, Qt-4.2.x, Qt-3.3.x, and Qt-3.2.x.
\item
  \NewSip{}.\\
  Supported versions: SIP-4.7.x, SIP-4.6.x, and SIP-4.5.x.\\
\item
  \PyQtGpl{}, \PyQtFGpl, \PyQtMac{}, \PyQtFMac{} and \PyQtCom{}.\\
  Supported versions: PyQt-4.3.x, PyQt-4.2.x, PyQt-4.1.x, and PyQt-3.17.x.\\
\item
  \QwtPlotDdd{}.\\
  Supported versions: \QwtPlotDddTgz{} or \QwtPlotDddZip{}.
  \PyQwtDddTarGz{} contains a version of \QwtPlotDdd{} for your convenience.
  You can (but unless you are using Windows do not have to) compile and link
  the \QwtPlotDdd{} sources statically into the PyQwt3D extension module.
\item
  \ZLib{} is needed to enable compression in the PDF and PostScript output of
  \QwtPlotDdd{}.\\
  Supported versions: zlib-1.2.x, and zlib-1.1.x.
  \PyQwtDddTarGz{} contains the necessary source files of zlib-1.2.3 to remove
  the dependency on zlib (but you are free to use an already installed shared
  or dynamic load library of zlib).
\end{enumerate}

\PrerequisitesEnd{}


\section{Installation\label{install}}

\Future{}

\subsection{Installation on \POSIX{} and MacOS/X\label{posix-install}}

\Future{}

The installation procedure consists of three steps:
\begin{enumerate}
\item
  Unpack \PyQwtDddTarGz{}.
\item
  Do a quick start to test the installation by running the commands:
\begin{verbatim}
cd PyQwt3D-0.1.6
cd configure
python configure.py -Q ../qwtplot3d-0.2.7
make
make install
\end{verbatim}
  where the directory
  \file{/sources/of/qwtplot3d} must contain the directories \file{3rdparty},
  \file{include} and \file{src}.\\
  The installation will fail if Qt has been configured without runtime type
  information (RTTI).  In this case, run the commands:
\begin{verbatim}
cd PyQwt3D-0.1.6
cd configure
python configure.py -Q /sources/of/qwtplot3d --extra-cxxflags=-frtti
make
make install
\end{verbatim}
  where \code{-frtti} enables RTTI for g++.  Check your compiler documention
  for other C++ compilers.
\item
  Fine tune (optional)
  \begin{itemize}
    \item
      to enable compression of PostScript and PDF files by running the
      commands:
\begin{verbatim}
python configure.py -Q /sources/of/qwtplot3d -l z -D HAVE_ZLIB
make
make install
\end{verbatim}
      Add
\begin{verbatim}
-L /directory/with/libz.*
\end{verbatim}
      to the \file{configure.py} options, if the linker fails to find the zlib
      library.
    \item
      to use a the QwtPlot3D library on your system by running the commands:
\begin{verbatim}
rm -rf Qwt3D
python configure.py -I /usr/include/qwtplot3d
make
make install
\end{verbatim}
      where \file{/usr/include/qwtplot3d} is an example for the installation
      directory of the QwtPlot3D header files.
      Add
\begin{verbatim}
-L /directory/with/libqwtplot3d.*
\end{verbatim}
      to the \file{configure.py} options, if the linker fails to find the
      QwtPlot3D library.
  \end{itemize}
\end{enumerate}

\begin{notice}[warning]
  The patched version of QwtPlot3D-0.2.7 included in PyQwt3D is required
  to enable saving of plots to the SVG and PGF vector file formats.
  In addition, you have to define HAVE_ZLIB and HAVE_LIBPNG to enable pixmaps
  in the SVG driver of GL2PS. A minimal configuration example is
\begin{verbatim}
python configure.py -Q ../qwtplot3d-0.2.7 -D HAVE_ZLIB -D HAVE_LIBPNG \
    -l z -l png
\end{verbatim}
  which may need extra options to indicate the location of the headers and
  libraries of libpng and zlib. For instance, a configuration example for
  Mac OS X with libpng from \MacPorts{} is:
\begin{verbatim}
python configure.py -Q ../qwtplot3d-O.2.7 -I /opt/local/include \
    -L /opt/local/lib -D HAVE_ZLIB -D HAVE_LIBPNG -l z -l png
\end{verbatim}
\end{notice}

\begin{notice}[note]
  \file{PyQwt-0.1.6/GNUmakefile} is makefile for GNU make which contains more
  examples of how to invoke \file{configure.py}.
  Adapt and use it, if you have GNU make.
\end{notice}

\begin{notice}[note]
  If you run into problems, send a log to the \mailinglist{}.

  There are at least two options to log the output of make:
  \begin{enumerate}
  \item Invoke make, tie stderr to stdout, and redirect stdout to LOG.txt:
\begin{verbatim}
# For Qt-3
make 3 2&>1 >LOG.txt
# For Qt-4
make 4 2&>1 >LOG.txt
\end{verbatim}
    However, you do not see what is going on.
  \item Use script to capture all screen output of make to LOG.txt:
\begin{verbatim}
# For Qt-3
script -c 'make 3' LOG.txt
# For Qt-4
script -c 'make 4' LOG.txt
\end{verbatim}
    The script command appeared in 3.0BSD and is part of util-linux.
  \end{enumerate}
\end{notice}

\begin{notice}[note]
  The configure.py script takes many options. The command
\begin{verbatim}
python configure.py -h
\end{verbatim}
  displays a full list of the available options:
  \verbatiminput{configure.help}
\end{notice}


\subsection{Installation on Windows with MSVC\label{win-install}}

\Future{}

The installation procedure consists of three steps:
\begin{enumerate}
\item
  Unpack \PyQwtDddTarGz{}.
\item
  Do a quick start to test the installation by running the commands:
\begin{verbatim}
cd PyQwt3D-0.1.6
cd configure
python configure.py -Q ..\qwtplot3d-0.2.7
nmake
nmake install
\end{verbatim}
  where the folder
  \file{C:\textbackslash{}sources\textbackslash{}of\textbackslash{}qwtplot3d}
  must contain the folders \file{3rdparty}, \file{include} and \file{src}.
  You can also edit one of the files \file{go3.bat}, \file{go4.bat}, or
  \file{go-mingw.bat} to suit your setup.
\item
  Fine tune (optional) by running the commands:
\begin{verbatim}
python configure.py -Q C:\sources\of\qwtplot3d -l zlib -D HAVE_ZLIB
nmake
nmake install
\end{verbatim}
    to enable compression of PostScript and PDF files. Add
\begin{verbatim}
-L C:\folder\containing\zlib.lib
\end{verbatim}
    to the \file{configure.py} options, if the linker fails to find the zlib
    library.
\end{enumerate}

\begin{notice}[note]
  The files \file{configure\textbackslash{}go3.bat} and
  \file{configure\textbackslash{}go4.bat} are examples of how to automatize
  the invokations of \strong{configure.py}, \strong{nmake}, and
  \strong{nmake install}.

  The file \file{configure\textbackslash{}go-mingw.bat} is used to build
  PyQwt3D with MinGW after building libpng by hand.

  Adapt and use one of those files.
\end{notice}

\begin{notice}[note]
  If you run into problems, send a log to the \mailinglist{}.

  Try
\begin{verbatim}
go3.bat >LOG.txt
\end{verbatim}
  or
\begin{verbatim}
go4.bat >LOG.txt
\end{verbatim}
  to make a log.
\end{notice}

\begin{notice}[note]
  The configure.py script takes many options. The command
\begin{verbatim}
python configure.py -h
\end{verbatim}
  displays a full list of the available options:
  \verbatiminput{configure.help}
\end{notice}

\begin{notice}[note]
  Since PyQwt3D wraps some classes and functions that are not exported from
  a QwtPlot3D dynamic load library, you have to compile and link the QwtPlot3D
  sources into PyQwt3D's extension module.
\end{notice}

\subsection{Binary Installer for Windows
  \label{windows-nsis}}

\Future{}

Run the following installers, if you have not done so:
\begin{enumerate}
\item \PythonExe{}.
\item \NumPyExe{}.
\item \QtNewExe{}.\\
  Let the Qt installer install MinGW if for instance \DevCpp{} has not been
  installed before Qt.
\item \PyQtNewExe{}.\\
  Make sure that you can run the PyQt examples by tweaking the environment
  variable \var{PATH}.
\item Install \PyQwtNewExe{}.
\end{enumerate}



\chapter{PyQwt3D Module Reference \label{reference}}

\Future{}

The reference should be used in conjunction with the \QwtPlotDddManual{}
and the \QwtPlotDddApi{}.
Only the differences specific to the Python bindings are documented here.

In this chapter, \emph{is not yet implemented} implies that the feature can
be easily implemented if needed, \emph{is not implemented} implies that the
feature is not easily implemented, and \emph{is not Pythonic} implies that
the feature will not be implemented because it violates the Python philosophy
(e.g. may use dangling pointers).

If a class is described as being \emph{fully implemented} then all non-private
member functions and all public class variables have been implemented.

Undocumented classes have not yet been implemented or are still experimental.

The classes in the QwtPlot3D library have quite a few protected attributes.
They are not easily exported to Python (SIP wraps protected member function,
but not protected attributes).
I will export protected attributes to Python on demand. For instance,
\var{Enrichment.plot} is accessible from Python, but protected in C++.


\section{Class reference \label{classes}}

\Future{}

\begin{classdesc*}{Arrow}
  is fully implemented.
\end{classdesc*}

\begin{classdesc*}{AutoScaler}
  is fully implemented.
\end{classdesc*}

\begin{classdesc*}{Axis}
  is fully implemented.
\end{classdesc*}

\begin{classdesc*}{AxisVector}
  wraps \ctype{std::vector<Axis>}. See \ref{wrappers} for details.
\end{classdesc*}

\begin{classdesc*}{Cell}
  wraps \ctype{std::vector<unsigned>}. See \ref{wrappers} for details.
\end{classdesc*}

\begin{classdesc*}{CellData}
  is fully implemented.
\end{classdesc*}

\begin{classdesc*}{CellField}
  wraps \ctype{std::vector<Cell>}. See \ref{wrappers} for details.
\end{classdesc*}

\begin{classdesc*}{Color}
  is fully implemented.
\end{classdesc*}

\begin{classdesc*}{ColorLegend}
  is fully implemented.
\end{classdesc*}

\begin{classdesc*}{Cone}
  is fully implemented.
\end{classdesc*}

\begin{classdesc*}{CoordinateSystem}
  is fully implemented.
\end{classdesc*}

\begin{classdesc*}{CrossHair}
  is fully implemented.
\end{classdesc*}

\begin{classdesc*}{Data}
  is fully implemented.\\
  FIXME: what to do with the protected data members?
\end{classdesc*}

\begin{classdesc*}{Dot}
  is fully implemented.
\end{classdesc*}

\begin{classdesc*}{DoubleVector}
  wraps \ctype{std::vector<double>}. See \ref{wrappers} for details.
\end{classdesc*}

\begin{classdesc*}{Drawable}
  is fully implemented.\\
  FIXME: what to do with the protected data members?
\end{classdesc*}

\begin{classdesc*}{Enrichment}
  is fully implemented.\\
  \begin{cvardesc}{const Plot3D*}{plot}
    This C++ protected data member is accessible in Python.
  \end{cvardesc}
\end{classdesc*}

\begin{classdesc*}{Freevector}
  is fully implemented.
\end{classdesc*}

\begin{classdesc*}{FreeVectorField}
  wraps \ctype{std::vector<FreeVector>}. See \ref{wrappers} for details.
\end{classdesc*}

\begin{classdesc*}{Function}
  is fully implemented.
\end{classdesc*}

\begin{classdesc*}{GLStateBewarer}
  is fully implemented.
\end{classdesc*}

\begin{classdesc*}{GridData}
  \begin{itemize}
  \item{vertices}. The public data member:
\begin{verbatim}
DataMatrix vertices;
\end{verbatim}
    is not accessible. FIXME: how to wrap \class{DataMatrix} safely?
  \item{normals}. The public data member:
\begin{verbatim}
DataMatrix normals;
\end{verbatim}
    is not accessible. FIXME: how to wrap \class{DataMatrix} safely?
  \end{itemize}
\end{classdesc*}

\begin{classdesc*}{GridMapping}
  is fully implemented.\\
  FIXME: what to do with the protected data members?
\end{classdesc*}

\begin{classdesc*}{IO}
  \begin{itemize}
  \item{defineInputHandler}. C++ declaration:
\begin{verbatim}
static bool defineInputHandler(QString const& format, Function func);
\end{verbatim}
    is not implemented (it is impossible to implement callbacks without an
    extra void pointer to hold a Python callable).
  \item{defineOutputHandler}. C++ declaration:
\begin{verbatim}
static bool defineOutputHandler(QString const& format, Function func);
\end{verbatim}
    is not implemented (it is impossible to implement callbacks without an
    extra void pointer to hold a Python callable).
  \end{itemize}
\end{classdesc*}

\begin{classdesc*}{Label}
  is fully implemented.
\end{classdesc*}

\begin{classdesc*}{LinearAutoscaler}
  is fully implemented.
\end{classdesc*}

\begin{classdesc*}{LinearScale}
  is fully implemented.\\
  FIXME: what to do with the protected data members?
\end{classdesc*}

\begin{classdesc*}{LogScale}
  is fully implemented.
\end{classdesc*}

\begin{classdesc*}{Mapping}
  is fully implemented.
\end{classdesc*}

\begin{classdesc*}{NativeReader}
  is fully implemented.
\end{classdesc*}

\begin{classdesc*}{ParallelEpiped}
  is fully implemented.
\end{classdesc*}

\begin{classdesc*}{ParametricSurface}
  is fully implemented.
\end{classdesc*}

\begin{classdesc*}{PixmapWriter}
  is fully implemented.
\end{classdesc*}

\begin{classdesc*}{Plot3D}
  is fully implemented.\\
  FIXME: what to do with the protected data members?\\
\end{classdesc*}

\begin{classdesc*}{RGBA}
  is fully implemented.
\end{classdesc*}

\begin{classdesc*}{Scale}
  is fully implemented.\\
  FIXME: what to do with the protected data members?
\end{classdesc*}

\begin{classdesc*}{StandardColor}
  is fully implemented.\\
  FIXME: what to do with the protected data members?
\end{classdesc*}

\begin{classdesc*}{SurfacePlot}
  is fully implemented.
  \begin{itemize}
  \item{facets}. C++ declaration:
\begin{verbatim}
std::pair<int,int> facets() const;
\end{verbatim}
    returns a tuple of two Python ints.
  \item{loadFromData}. C++ declaration:
\begin{verbatim}
bool loadFromData(Qwt3D::Triple** data,
                  unsigned int columns, unsigned int rows,
                  bool uperiodic = false, bool vperiodic = false);
\end{verbatim}
    is wrapped by:
\begin{verbatim}
success = surfacePlot.loadFromData(data, uperiodic = False, vperiodic = False)
\end{verbatim}
    where \var{success} is \constant{True} or \constant{False}, \var{data}
    is convertable to a Numeric or numarray array of Python floats with a shape
    (N, M, 3), and \var{uperiodic} and \var{vperiodic} are Python bools.\\
    C++ declaration:
\begin{verbatim}
bool loadFromData(double** data, unsigned int columns, unsigned int rows,
                  double minx, double maxx, double miny, double maxy);
\end{verbatim}
    is wrapped by:
\begin{verbatim}
success = surfacePlot.loadFromData(data, minx, maxx, miny, maxy)
\end{verbatim}
    where \var{success} is \constant{True} or \constant{False}, \var{data}
    is convertable to a Numeric or numarray array of Python floats with a shape
    (N, M), and \var{minx}, \var{maxx}, \var{miny} and \var{maxy} are
    convertable to Python floats.\\
    C++ declaration:
\begin{verbatim}
bool loadFromData(Qwt3D::TripleField const& data,
                  Qwt3D::CellField const& poly);
\end{verbatim}
    is wrapped by:
\begin{verbatim}
success = surfacePlot.loadFromData(tripleField, cellField)
\end{verbatim}
    where \var{success} is \constant{True} or \constant{False},
    \var{tripleField} is a \class{TripleField}, and \var{cellField} is a
    \class{CellField}.
  \item{createDataRepresentation}. C++ declarations:
\begin{verbatim}
bool createDataRepresentation(
     Qwt3D::Triple** data, unsigned int columns, unsigned int rows,
     bool uperiodic = false, bool vperiodic = false);
bool createDataRepresentation(
     double** data, unsigned int columns, unsigned int rows,
     double minx, double maxx, double miny, double maxy);
bool createDataRepresentation(
     Qwt3D::TripleField const& data, Qwt3D::CellFieldconst& poly)
\end{verbatim}
    are deprecated and therefore not implemented.
  \item{readIn}. C++ declaration:
\begin{verbatim}
void readIn(Qwt3D::GridData& grid, Triple** data,
            unsigned int columns, unsigned int rows);
\end{verbatim}
    is wrapped by:
\begin{verbatim}
surfacePlot.readIn(gridData, data) 
\end{verbatim}
    where \var{gridData} is a \class{GridData}, and \var{data} is convertable
    to a Numeric or numarray array of Python floats with a shape (N, M, 3).\\
    C++ declaration:
\begin{verbatim}
void readIn(Qwt3D::GridData& grid, double** data,
            unsigned int columns, unsigned int rows,
            double minx, double maxx, double miny, double maxy);
\end{verbatim}
    is wrapped by:
\begin{verbatim}
surfacePlot.readIn(gridData, data, minx, maxx, miny, maxy)
\end{verbatim}
    where \var{gridData} is a \class{GridData}, \var{data} is convertable to
    a Numeric or numarray array of Python floats with a shape (N, M), and
    \var{minx}, \var{maxx}, \var{miny} and \var{maxy} are convertable to Python
    floats.
  \end{itemize}
\end{classdesc*}

\begin{classdesc*}{Triple}
  \begin{itemize}
  \item{operator *()}. C++ declaration:
\begin{verbatim}
Triple operator*(double, const Triple &);
\end{verbatim}
    is not implemented.
  \item{operator /()}. C++ declaration:
\begin{verbatim}
Triple operator/(double, const Triple &);
\end{verbatim}
    is not implemented.
  \end{itemize}
\end{classdesc*}

\begin{classdesc*}{TripleField}
wraps \ctype{std::vector<Triple>}. See \ref{wrappers} for details.
\end{classdesc*}

\begin{classdesc*}{Tuple}
is fully implemented.
\end{classdesc*}

\begin{classdesc*}{VectorWriter}
is fully implemented.
\end{classdesc*}

\begin{classdesc*}{VertexEnrichment}
is fully implemented.
\end{classdesc*}

\section{Wrappers for \ctype{std::vector<T>} \label{wrappers}}

\Future{}

PyQwt3D has a partial interface to the following C++ std::vector templates:
\begin{enumerate}
\item
  \class{AxisVector} for \ctype{std::vector<Axis>}
\item
  \class{Cell} for \ctype{std::vector<unsigned>}
\item
  \class{CellField} for \ctype{std::vector<Cell>}
\item
  \class{ColorVector} for \ctype{std::vector<RGBA>}
\item
  \class{DoubleVector} for \ctype{std::vector<double>}
\item
  \class{FreeVectorField} for \ctype{std::vector<FreeVectorField>}
\item
  \class{TripleField} for \ctype{std::vector<Triple>}
\end{enumerate}

The interface implements four constructors for each template instantianation --
taking Cell as example:
\begin{enumerate}
\item
  \code{Cell()}
\item
  \code{Cell(size)}
\item
  \code{Cell(size, item)}
\item
  \code{Cell(otherCell)}
\end{enumerate}

and 13 member functions -- taking Cell as example:
\begin{enumerate}
\item
  \code{result = cell.capacity()}
\item
  \code{cell.clear()}
\item
  \code{result = cell.empty()}
\item
  \code{result = cell.back()}
\item
  \code{result = cell.front()}
\item
  \code{result = cell.max_size()}
\item
  \code{cell.pop_back()}
\item
  \code{cell.push_back(item)}
\item
  \code{cell.reserve(size)}
\item
  \code{cell.reserve(size, item = 0)}
\item
  \code{cell.resize(size, item = 0)}
\item
  \code{result = cell.size()}
\item
  \code{cell.swap(otherCell)}
\end{enumerate}

Iterators are not yet implemented. However, the implementation of the
Python slots \function{__getitem__}, \function{__len__} and
\function{__setitem__} let you use those classes almost as a sequence.
For instance:

\verbatiminput{StdVectorExample.txt}

\section{Function reference \label{functions}}

\Future{}

\begin{funcdesc}{plot}{x, y, function, title='', labels=('x', 'y', 'z')}
  Returns a plot of \code{function(x, y)}
  
  Here, \var{x} and \var{y} are vectors, \var{function} is function of two
  variables, \var{title} as string and \var{labels} a sequence of three strings.

  The axes are scaled to make the axis frame cubic.
\end{funcdesc}

\begin{funcdesc}{save}{plot3d, name, format,
    landscape=VectorWriter.OFF,
    textmode=VectorWriter.NATIVE,
    sortmode=sortmode=VectorWriter.BSPSORT}
  Saves a snapshot of a Plot3D widget to a file.

  Here, \var{plot3d} is a Plot3D  widget or a widget with a Plot3D widget as
  child, \var{name} is the file name, and \var{format} a case-insensitive
  string indicating the file format.  \var{landscape} can be
  \constant{VectorWriter.ON}, \constant{VectorWriter.OFF}, or
  \constant{VectorWriter.AUTO}, \var{textmode} can be
  \constant{VectorWriter.PIXEL}, \constant{VectorWriter.NATIVE}, or
  \constant{VectorWriter.TEX}, and \var{sortmode} can be
  \constant{VectorWriter.NOSORT}, \constant{VectorWriter.SIMPLESORT}, or
  \constant{VectorWriter.BSPSORT}.

  PyQwt3D uses GL2PS to support vector formats as EPS, EPS_GZ, PDF, PGF, PS,
  PS_GZ, SVG, and SVG_GZ. It uses Qt to support pixmap formats as GIF, JPEG,
  PNG, and others.
    
  \function{save} returns \constant{True} on success and \constant{False} on
  failure.
\end{funcdesc}


\begin{cfuncdesc}{const GLubyte*}{gl_error}{}
  is implemented as
  \begin{verbatim}
message = gl_error()
  \end{verbatim}
\end{cfuncdesc}

\begin{cfuncdesc}{bool}{ViewPort2World}
  {double \&wx, double \&wy, double \&wz, double vx, double vy, double vz}
  is implemented as
  \begin{verbatim}
success, wx, wy, wz = ViewPort2World(vx, vy, vz)
  \end{verbatim}
\end{cfuncdesc}

\begin{cfuncdesc}{bool}{World2ViewPort}
  {double \&vx, double \&vy, double \&vz, double wx, double wy, double wz}
  is implemented as
  \begin{verbatim}
success, vx, vy, vz = World2Viewport(wx, wy, wz)
  \end{verbatim}
\end{cfuncdesc}


\documentclass{manual}

% Links
\newcommand{\QwtPlotDdd}{\ulink{QwtPlot3D}
  {http://qwtplot3d.sourceforge.net}}
\newcommand{\QwtPlotDddApi}{\ulink{QwtPlot3D API documentation}
  {http://qwtplot3d.sourceforge.net/web/navigation/api_frame.html}}
\newcommand{\QwtPlotDddManual}{\ulink{QwtPlot3D manual}
  {http://qwtplot3d.sourceforge.net/web/navigation/manual_frame.html}}
\newcommand{\ZLib}{\ulink{ZLib}
  {http://www.gzip.org/zlib}}
\newcommand{\mailinglist}{\ulink{mailing list}
  {mailto:pyqwt-users@lists.sourceforge.net}}

% Source code
\newcommand{\snapshot}{\ulink{snapshot}
  {http://www.river-bank.demon.co.uk/download/snapshots}}
\newcommand{\NumPy}{\ulink{NumPy}
  {http://www.numpy.org}}
\newcommand{\NumPyTarGz}{\ulink{numpy-1.0rc1}
  {http://prdownloads.sourceforge.net/numpy/numpy-1.0rc1.tar.gz?download}}
\newcommand{\numarray}{\ulink{numarray}
  {http://www.stsci.edu/resources/software_hardware/numarray}}
\newcommand{\numarrayTarGz}{\ulink{numarray-1.5.2.tar.gz}
  {http://prdownloads.sourceforge.net/numpy/numarray-1.5.2.tar.gz?download}}
\newcommand{\Numeric}{\ulink{Numeric}
  {http://www.numpy.org}}
\newcommand{\NumericTarGz}{\ulink{Numeric-24.2.tar.gz}
  {http://prdownloads.sourceforge.net/numpy/Numeric-24.2.tar.gz?download}}
\newcommand{\NewSip}{\ulink{sip-4.4.5.tar.gz}
  {http://pyqwt.sourceforge.net/support/sip-4.4.5.tar.gz}}
\newcommand{\SipSnapShot}{\ulink{SIP snapshot}
  {http://www.riverbankcomputing.com/Downloads/Snapshots/sip4/}}
\newcommand{\PyQtGpl}{\ulink{PyQt-x11-gpl-3.16.tar.gz}
  {http://pyqwt.sourceforge.net/support/PyQt-x11-gpl-3.16.tar.gz}}
\newcommand{\PyQtMac}{\ulink{PyQt-mac-gpl-3.16.tar.gz}
  {http://pyqwt.sourceforge.net/support/PyQt-mac-gpl-3.16.tar.gz}}
\newcommand{\PyQtSnapShot}{\ulink{PyQt3 snapshot}
  {http://www.riverbankcomputing.com/Downloads/Snapshots/PyQt3}}
\newcommand{\PyQtFSnapShot}{\ulink{PyQt4 snapshot}
  {http://www.riverbankcomputing.com/Downloads/Snapshots/PyQt4}}
\newcommand{\PyQtFGpl}{\ulink{PyQt4-x11-gpl-4.0.1.tar.gz}
  {http://pyqwt.sourceforge.net/support/PyQt-x11-gpl-4.0.1.tar.gz}}
\newcommand{\PyQtFMac}{\ulink{PyQt4-mac-gpl-4.0.1.tar.gz}
  {http://pyqwt.sourceforge.net/support/PyQt-mac-gpl-4.0.1.tar.gz}}
\newcommand{\PyQtCom}{\ulink{PyQt-commercial}
  {http://www.riverbankcomputing.co.uk/pyqt/buy.php}}
\newcommand{\PyQwtDddTarGz}{\ulink{PyQwt3D-0.1.2.tar.gz}
  {http://prdownloads.sourceforge.net/pyqwt/PyQwt3D-0.1.2.tar.gz?download}}
\newcommand{\QwtPlotDddTgz}{\ulink{qwtplot3d-0.2.6.tgz}
  {http://prdownloads.sourceforge.net/qwtplot3d/qwtplot3d-0.2.6.tgz?download}}
\newcommand{\QwtPlotDddZip}{\ulink{qwtplot3d-0.2.6.zip}
  {http://prdownloads.sourceforge.net/qwtplot3d/qwtplot3d-0.2.6.zip?download}}

% Installers for MS-Windows
\newcommand{\PythonMsi}{\ulink{python-2.4.2/msi}
  {http://python.org/ftp/python/2.4/python-2.4.2.msi}}
\newcommand{\NumericExe}{\ulink{Numeric-24.2.win32-py2.4.exe}
  {http://prdownloads.sourceforge.net/numpy/Numeric-24.2.win32-py2.4.exe?download}}
\newcommand{\numarrayExe}{\ulink{numarray-1.5.2.win32-py2.4.exe}
  {http://prdownloads.sourceforge.net/numpy/numarray-1.5.2.win32-py2.4.exe?download}}
\newcommand{\NumPyExe}{\ulink{numpy-1.0.0.win32-py2.4.exe}
  {http://prdownloads.sourceforge.net/numpy/numpy-1.0.0.win32-py2.4.exe?download}}


\newcommand{\PrerequisitesEnd}{
  To exploit the full power of the PyQwt3D, you should install at
  least one the Numerical Python extensions:
  \NumPy{}, \numarray{}, \Numeric{}.
  I am using \NumPyTarGz{}, \NumericTarGz{} and \numarrayTarGz{}.

  \begin{notice}[note]
    The PyQwt3D extension module containing statically linked sources of
    QwtPlot3D-0.2.6 coexists very well with system wide shared libraries
    of any version of QwtPlot3D.
  \end{notice}

  \begin{notice}[warning]
    PyQwt3D may not work with numarray on Linux systems, possibly due to a bug
    in the floating point excepion handling of glibc. 
    More information is to be found
    \ulink{here}{http://sourceforge.net/mailarchive/message.php?msg_id=9914816}
    and in related posts.
    PyQwt3D does not work with numarray and Matrox graphic cards on
    Mandrake-10.0 and SuSE-9.0.
    Your mileage may vary: PyQwt3D works with numarray and an ATI graphic card
    on SuSE-9.1.
  \end{notice}
}

\newcommand{\Future}{
  \begin{notice}[warning]
    The documentation is for the future PyQwt3D-0.1.2 which is only available
    from CVS.
  \end{notice}
}

%\renewcommand{\Future}{}


\title{PyQwt3D Manual}

% boilerplate.tex?
\author{Gerard Vermeulen}

\date{\today}
\release{0.1.2}
\setshortversion{0.1.2}

\makeindex

\begin{document}

\maketitle

% This makes the contents more accessible from the front page of the HTML.
\ifhtml
\chapter*{Front Matter \label{front}}
\fi

Copyright \copyright{} 2004 Gerard Vermeulen

PyQwt3D is free software; you can redistribute it and/or modify it under the
terms of the GNU General Public License as published by the Free Software
Foundation; either version 2 of the License, or (at your option) any later
version.

PyQwt3D is distributed in the hope that it will be useful, but WITHOUT ANY
WARRANTY; without even the implied warranty of MERCHANTABILITY or FITNESS
FOR A PARTICULAR PURPOSE.  See the GNU  General Public License for more
details.

You should have received a copy of the GNU General Public License along with
PyQwt3D; if not, write to the Free Software Foundation, Inc., 59 Temple Place,
Suite 330, Boston, MA 02111-1307, USA.

In addition, as a special exception, Gerard Vermeulen gives permission to
link PyQwt3D dynamically with commercial, non-commercial or educational 
versions of Qt, PyQt and sip, and distribute PyQwt3D in this form, provided
that equally powerful versions of Qt, PyQt and sip have been released under
the terms of the GNU General Public License.

If PyQwt3D is dynamically linked with commercial, non-commercial or
educational versions of Qt, PyQt and sip, PyQwt3D becomes a free plug-in
for a non-free program.



\begin{abstract}

\noindent
PyQwt3D is a set of Python bindings for the \QwtPlotDdd{} library which extends
the Qt framework with widgets to visualize 3-dimensional data.
It allows you to integrate PyQt, Qt, QwtPlot3D, the Numerical Python
extensions, and optionally SciPy in a GUI Python application or in an
interactive Python session.

\end{abstract}

\tableofcontents

\chapter{Introduction\label{introduction}}

\Future{}

PyQwt3D is a set of Python bindings for the \QwtPlotDdd{} library which extends
the Qt framework with widgets to visualize 3-dimensional data.


\chapter{Installation\label{installation}}

\Future{}

\begin{notice}[note]
  PyQwt3D for Qt-3.3.x can coexist with PyQwt3D for Qt-4.1.x. The statement
  \begin{verbatim}
import Qwt3D
  \end{verbatim}
  imports PyQwt3D for Qt-3.3.x and the statement
  \begin{verbatim}
import PyQt4.Qwt3D
  \end{verbatim}
  imports PyQwt3D for Qt-4.1.x. 
\end{notice}


\section{Installation prerequisites\label{prerequisites}}

\Future{}

Installation prerequisites for \PyQwtDddTarGz{} are:
\begin{enumerate}
\item
  \ulink{Python}{http://www.python.org}.\\
  Supported versions: Python-2.5.x, Python-2.4.x and Python-2.3.x.
\item
  \ulink{Qt}{http://www.trolltech.com}.\\
  Supported versions: Qt-4.2.x, Qt-4.1.x, Qt-3.3.x, or Qt-3.2.x.
\item
  \NewSip{}.\\
  Supported versions: SIP-4.4.x.\\
  Python-2.5 requires a \SipSnapShot{}.\\
\item
  \PyQtGpl{}, \PyQtFGpl, \PyQtMac{}, \PyQtFMac{} or \PyQtCom{}.\\
  Supported versions: PyQt-4.0.x, or PyQt-3.16.x.\\
  Python-2.5 requires a \PyQtSnapShot{} or \PyQtFSnapShot{}.\\
\item
  \QwtPlotDdd{}.\\
  Supported versions: \QwtPlotDddTgz{} or \QwtPlotDddZip{}.
  \PyQwtDddTarGz{} contains a version of \QwtPlotDdd{} for your convenience.
  You can (but unless you are using Windows do not have to) compile and link
  the \QwtPlotDdd{} sources statically into the PyQwt3D extension module.
\item
  \ZLib{} is needed to enable compression in the PDF and PostScript output of
  \QwtPlotDdd{}.\\
  Supported versions: zlib-1.2.x, and zlib-1.1.x.
  \PyQwtDddTarGz{} contains the necessary source files of zlib-1.2.3 to remove
  the dependency on zlib (but you are free to use an already installed shared
  or dynamic load library of zlib).
\end{enumerate}

\PrerequisitesEnd{}


\section{Installation\label{install}}

\Future{}

\subsection{Installation on \POSIX{} and MacOS/X\label{posix-install}}

\Future{}

The installation procedure consists of three steps:
\begin{enumerate}
\item
  Unpack \PyQwtDddTarGz{}.
\item
  Do a quick start to test the installation by running the commands:
\begin{verbatim}
cd PyQwt3D-0.1.2
cd configure
python configure.py -Q ../qwtplot3d-0.2.6
make
make install
\end{verbatim}
  where the directory
  \file{/sources/of/qwtplot3d} must contain the directories \file{3rdparty},
  \file{include} and \file{src}.\\
  The installation will fail if Qt has been configured without runtime type
  information (RTTI).  In this case, run the commands:
\begin{verbatim}
cd PyQwt3D-0.1.2
cd configure
python configure.py -Q /sources/of/qwtplot3d --extra-cxxflags=-frtti
make
make install
\end{verbatim}
  where \code{-frtti} enables RTTI for g++.  Check your compiler documention
  for other C++ compilers.
\item
  Fine tune (optional)
  \begin{itemize}
    \item
      to enable compression of PostScript and PDF files by running the
      commands:
\begin{verbatim}
python configure.py -Q /sources/of/qwtplot3d -l z -D GL2PS_HAVE_ZLIB
make
make install
\end{verbatim}
      Add
\begin{verbatim}
-L /directory/with/libz.*
\end{verbatim}
      to the \file{configure.py} options, if the linker fails to find the zlib
      library.
    \item
      to use a the QwtPlot3D library on your system by running the commands:
\begin{verbatim}
rm -rf Qwt3D
python configure.py -I /usr/include/qwtplot3d
make
make install
\end{verbatim}
      where \file{/usr/include/qwtplot3d} is an example for the installation
      directory of the QwtPlot3D header files.
      Add
\begin{verbatim}
-L /directory/with/libqwtplot3d.*
\end{verbatim}
      to the \file{configure.py} options, if the linker fails to find the
      QwtPlot3D library.
  \end{itemize}
\end{enumerate}

\begin{notice}[note]
  \file{PyQwt-0.1.2/GNUmakefile} is makefile for GNU make which contains more
  examples of how to invoke \file{configure.py}.
  Adapt and use it, if you have GNU make.
\end{notice}

\begin{notice}[note]
  If you run into problems, send a log to the \mailinglist{}.

  There are at least two options to log the output of make:
  \begin{enumerate}
  \item Invoke make, tie stderr to stdout, and redirect stdout to LOG.txt:
\begin{verbatim}
# For Qt-3
make 3 2&>1 >LOG.txt
# For Qt-4
make 4 2&>1 >LOG.txt
\end{verbatim}
    However, you do not see what is going on.
  \item Use script to capture all screen output of make to LOG.txt:
\begin{verbatim}
# For Qt-3
script -c 'make 3' LOG.txt
# For Qt-4
script -c 'make 4' LOG.txt
\end{verbatim}
    The script command appeared in 3.0BSD and is part of util-linux.
  \end{enumerate}
\end{notice}

\begin{notice}[note]
  The configure.py script takes many options. The command
\begin{verbatim}
python configure.py -h
\end{verbatim}
  displays a full list of the available options:
  \verbatiminput{configure.help}
\end{notice}


\subsection{Installation on Windows with MSVC\label{win-install}}

\Future{}

The installation procedure consists of three steps:
\begin{enumerate}
\item
  Unpack \PyQwtDddTarGz{}.
\item
  Do a quick start to test the installation by running the commands:
\begin{verbatim}
cd PyQwt3D-0.1.2
cd configure
python configure.py -Q ..\qwtplot3d-0.2.6
nmake
nmake install
\end{verbatim}
  where the folder
  \file{C:\textbackslash{}sources\textbackslash{}of\textbackslash{}qwtplot3d}
  must contain the folders \file{3rdparty}, \file{include} and \file{src}.
  You can also edit the files \file{go3.bat} or \file{go4.bat} to suit your
  setup.
\item
  Fine tune (optional) by running the commands:
\begin{verbatim}
python configure.py -Q C:\sources\of\qwtplot3d -l zlib -D GL2PS_HAVE_ZLIB
nmake
nmake install
\end{verbatim}
    to enable compression of PostScript and PDF files. Add
\begin{verbatim}
-L C:\folder\containing\zlib.lib
\end{verbatim}
    to the \file{configure.py} options, if the linker fails to find the zlib
    library.
\end{enumerate}

\begin{notice}[note]
  The files \file{configure\textbackslash{}go3.bat} and
  \file{configure\textbackslash{}go4.bat} are examples of how to automatize
  the invokations of \strong{configure.py}, \strong{nmake}, and
  \strong{nmake install}.  Adapt and use it.
\end{notice}

\begin{notice}[note]
  If you run into problems, send a log to the \mailinglist{}.

  Try
\begin{verbatim}
go3.bat >LOG.txt
\end{verbatim}
  or
\begin{verbatim}
go4.bat >LOG.txt
\end{verbatim}
  to make a log.
\end{notice}

\begin{notice}[note]
  The configure.py script takes many options. The command
\begin{verbatim}
python configure.py -h
\end{verbatim}
  displays a full list of the available options:
  \verbatiminput{configure.help}
\end{notice}

\begin{notice}[note]
  Since PyQwt3D wraps some classes and functions that are not exported from
  a QwtPlot3D dynamic load library, you have to compile and link the QwtPlot3D
  sources into PyQwt3D's extension module.
\end{notice}


\chapter{PyQwt3D Module Reference \label{reference}}

\Future{}

The reference should be used in conjunction with the \QwtPlotDddManual{}
and the \QwtPlotDddApi{}.
Only the differences specific to the Python bindings are documented here.

In this chapter, \emph{is not yet implemented} implies that the feature can
be easily implemented if needed, \emph{is not implemented} implies that the
feature is not easily implemented, and \emph{is not Pythonic} implies that
the feature will not be implemented because it violates the Python philosophy
(e.g. may use dangling pointers).

If a class is described as being \emph{fully implemented} then all non-private
member functions and all public class variables have been implemented.

Undocumented classes have not yet been implemented or are still experimental.

The classes in the QwtPlot3D library have quite a few protected attributes.
They are not easily exported to Python (SIP wraps protected member function,
but not protected attributes).
I will export protected attributes to Python on demand. For instance,
\var{Enrichment.plot} is accessible from Python, but protected in C++.




\section{Class reference \label{classes}}

\Future{}

\begin{classdesc*}{Arrow}
  is fully implemented.
\end{classdesc*}

\begin{classdesc*}{AutoScaler}
  is fully implemented.
\end{classdesc*}

\begin{classdesc*}{Axis}
  is fully implemented.
\end{classdesc*}

\begin{classdesc*}{AxisVector}
  wraps \ctype{std::vector<Axis>}. See \ref{wrappers} for details.
\end{classdesc*}

\begin{classdesc*}{Cell}
  wraps \ctype{std::vector<unsigned>}. See \ref{wrappers} for details.
\end{classdesc*}

\begin{classdesc*}{CellData}
  is fully implemented.
\end{classdesc*}

\begin{classdesc*}{CellField}
  wraps \ctype{std::vector<Cell>}. See \ref{wrappers} for details.
\end{classdesc*}

\begin{classdesc*}{Color}
  is fully implemented.
\end{classdesc*}

\begin{classdesc*}{ColorLegend}
  is fully implemented.
\end{classdesc*}

\begin{classdesc*}{Cone}
  is fully implemented.
\end{classdesc*}

\begin{classdesc*}{CoordinateSystem}
  is fully implemented.
\end{classdesc*}

\begin{classdesc*}{CrossHair}
  is fully implemented.
\end{classdesc*}

\begin{classdesc*}{Data}
  is fully implemented.\\
  FIXME: what to do with the protected data members?
\end{classdesc*}

\begin{classdesc*}{Dot}
  is fully implemented.
\end{classdesc*}

\begin{classdesc*}{DoubleVector}
  wraps \ctype{std::vector<double>}. See \ref{wrappers} for details.
\end{classdesc*}

\begin{classdesc*}{Drawable}
  is fully implemented.\\
  FIXME: what to do with the protected data members?
\end{classdesc*}

\begin{classdesc*}{Enrichment}
  is fully implemented.\\
  \begin{cvardesc}{const Plot3D*}{plot}
    This C++ protected data member is accessible in Python.
  \end{cvardesc}
\end{classdesc*}

\begin{classdesc*}{Freevector}
  is fully implemented.
\end{classdesc*}

\begin{classdesc*}{FreeVectorField}
  wraps \ctype{std::vector<FreeVector>}. See \ref{wrappers} for details.
\end{classdesc*}

\begin{classdesc*}{Function}
  is fully implemented.
\end{classdesc*}

\begin{classdesc*}{GLStateBewarer}
  is fully implemented.
\end{classdesc*}

\begin{classdesc*}{GridData}
  \begin{itemize}
  \item{vertices}. The public data member:
\begin{verbatim}
DataMatrix vertices;
\end{verbatim}
    is not accessible. FIXME: how to wrap \class{DataMatrix} safely?
  \item{normals}. The public data member:
\begin{verbatim}
DataMatrix normals;
\end{verbatim}
    is not accessible. FIXME: how to wrap \class{DataMatrix} safely?
  \end{itemize}
\end{classdesc*}

\begin{classdesc*}{GridMapping}
  is fully implemented.\\
  FIXME: what to do with the protected data members?
\end{classdesc*}

\begin{classdesc*}{IO}
  \begin{itemize}
  \item{defineInputHandler}. C++ declaration:
\begin{verbatim}
static bool defineInputHandler(QString const& format, Function func);
\end{verbatim}
    is not implemented (it is impossible to implement callbacks without an
    extra void pointer to hold a Python callable).
  \item{defineOutputHandler}. C++ declaration:
\begin{verbatim}
static bool defineOutputHandler(QString const& format, Function func);
\end{verbatim}
    is not implemented (it is impossible to implement callbacks without an
    extra void pointer to hold a Python callable).
  \end{itemize}
\end{classdesc*}

\begin{classdesc*}{Label}
  is fully implemented.
\end{classdesc*}

\begin{classdesc*}{LinearAutoscaler}
  is fully implemented.
\end{classdesc*}

\begin{classdesc*}{LinearScale}
  is fully implemented.\\
  FIXME: what to do with the protected data members?
\end{classdesc*}

\begin{classdesc*}{LogScale}
  is fully implemented.
\end{classdesc*}

\begin{classdesc*}{Mapping}
  is fully implemented.
\end{classdesc*}

\begin{classdesc*}{NativeReader}
  is fully implemented.
\end{classdesc*}

\begin{classdesc*}{ParallelEpiped}
  is fully implemented.
\end{classdesc*}

\begin{classdesc*}{ParametricSurface}
  is fully implemented.
\end{classdesc*}

\begin{classdesc*}{PixmapWriter}
  is fully implemented.
\end{classdesc*}

\begin{classdesc*}{Plot3D}
  is fully implemented.\\
  FIXME: what to do with the protected data members?\\
\end{classdesc*}

\begin{classdesc*}{RGBA}
  is fully implemented.
\end{classdesc*}

\begin{classdesc*}{Scale}
  is fully implemented.\\
  FIXME: what to do with the protected data members?
\end{classdesc*}

\begin{classdesc*}{StandardColor}
  is fully implemented.\\
  FIXME: what to do with the protected data members?
\end{classdesc*}

\begin{classdesc*}{SurfacePlot}
  is fully implemented.
  \begin{itemize}
  \item{facets}. C++ declaration:
\begin{verbatim}
std::pair<int,int> facets() const;
\end{verbatim}
    returns a tuple of two Python ints.
  \item{loadFromData}. C++ declaration:
\begin{verbatim}
bool loadFromData(Qwt3D::Triple** data,
                  unsigned int columns, unsigned int rows,
                  bool uperiodic = false, bool vperiodic = false);
\end{verbatim}
    is wrapped by:
\begin{verbatim}
success = surfacePlot.loadFromData(data, uperiodic = False, vperiodic = False)
\end{verbatim}
    where \var{success} is \constant{True} or \constant{False}, \var{data}
    is convertable to a Numeric or numarray array of Python floats with a shape
    (N, M, 3), and \var{uperiodic} and \var{vperiodic} are Python bools.\\
    C++ declaration:
\begin{verbatim}
bool loadFromData(double** data, unsigned int columns, unsigned int rows,
                  double minx, double maxx, double miny, double maxy);
\end{verbatim}
    is wrapped by:
\begin{verbatim}
success = surfacePlot.loadFromData(data, minx, maxx, miny, maxy)
\end{verbatim}
    where \var{success} is \constant{True} or \constant{False}, \var{data}
    is convertable to a Numeric or numarray array of Python floats with a shape
    (N, M), and \var{minx}, \var{maxx}, \var{miny} and \var{maxy} are
    convertable to Python floats.\\
    C++ declaration:
\begin{verbatim}
bool loadFromData(Qwt3D::TripleField const& data,
                  Qwt3D::CellField const& poly);
\end{verbatim}
    is wrapped by:
\begin{verbatim}
success = surfacePlot.loadFromData(tripleField, cellField)
\end{verbatim}
    where \var{success} is \constant{True} or \constant{False},
    \var{tripleField} is a \class{TripleField}, and \var{cellField} is a
    \class{CellField}.
  \item{createDataRepresentation}. C++ declarations:
\begin{verbatim}
bool createDataRepresentation(
     Qwt3D::Triple** data, unsigned int columns, unsigned int rows,
     bool uperiodic = false, bool vperiodic = false);
bool createDataRepresentation(
     double** data, unsigned int columns, unsigned int rows,
     double minx, double maxx, double miny, double maxy);
bool createDataRepresentation(
     Qwt3D::TripleField const& data, Qwt3D::CellFieldconst& poly)
\end{verbatim}
    are deprecated and therefore not implemented.
  \item{readIn}. C++ declaration:
\begin{verbatim}
void readIn(Qwt3D::GridData& grid, Triple** data,
            unsigned int columns, unsigned int rows);
\end{verbatim}
    is wrapped by:
\begin{verbatim}
surfacePlot.readIn(gridData, data) 
\end{verbatim}
    where \var{gridData} is a \class{GridData}, and \var{data} is convertable
    to a Numeric or numarray array of Python floats with a shape (N, M, 3).\\
    C++ declaration:
\begin{verbatim}
void readIn(Qwt3D::GridData& grid, double** data,
            unsigned int columns, unsigned int rows,
            double minx, double maxx, double miny, double maxy);
\end{verbatim}
    is wrapped by:
\begin{verbatim}
surfacePlot.readIn(gridData, data, minx, maxx, miny, maxy)
\end{verbatim}
    where \var{gridData} is a \class{GridData}, \var{data} is convertable to
    a Numeric or numarray array of Python floats with a shape (N, M), and
    \var{minx}, \var{maxx}, \var{miny} and \var{maxy} are convertable to Python
    floats.
  \end{itemize}
\end{classdesc*}

\begin{classdesc*}{Triple}
  \begin{itemize}
  \item{operator *()}. C++ declaration:
\begin{verbatim}
Triple operator*(double, const Triple &);
\end{verbatim}
    is not implemented (not supported by SIP-4.2.x).
  \item{operator /()}. C++ declaration:
\begin{verbatim}
Triple operator/(double, const Triple &);
\end{verbatim}
    is not implemented (not supported by SIP-4.2.x).
  \end{itemize}
\end{classdesc*}

\begin{classdesc*}{TripleField}
wraps \ctype{std::vector<Triple>}. See \ref{wrappers} for details.
\end{classdesc*}

\begin{classdesc*}{Tuple}
is fully implemented.
\end{classdesc*}

\begin{classdesc*}{VectorWriter}
is fully implemented.
\end{classdesc*}

\begin{classdesc*}{VertexEnrichment}
is fully implemented.
\end{classdesc*}

\section{Wrappers for \ctype{std::vector<T>} \label{wrappers}}

\Future{}

PyQwt3D has a partial interface to the following C++ std::vector templates:
\begin{enumerate}
\item
  \class{AxisVector} for \ctype{std::vector<Axis>}
\item
  \class{Cell} for \ctype{std::vector<unsigned>}
\item
  \class{CellField} for \ctype{std::vector<Cell>}
\item
  \class{ColorVector} for \ctype{std::vector<RGBA>}
\item
  \class{DoubleVector} for \ctype{std::vector<double>}
\item
  \class{FreeVectorField} for \ctype{std::vector<FreeVectorField>}
\item
  \class{TripleField} for \ctype{std::vector<Triple>}
\end{enumerate}

The interface implements four constructors for each template instantianation --
taking Cell as example:
\begin{enumerate}
\item
  \code{Cell()}
\item
  \code{Cell(size)}
\item
  \code{Cell(size, item)}
\item
  \code{Cell(otherCell)}
\end{enumerate}

and 13 member functions -- taking Cell as example:
\begin{enumerate}
\item
  \code{result = cell.capacity()}
\item
  \code{cell.clear()}
\item
  \code{result = cell.empty()}
\item
  \code{result = cell.back()}
\item
  \code{result = cell.front()}
\item
  \code{result = cell.max_size()}
\item
  \code{cell.pop_back()}
\item
  \code{cell.push_back(item)}
\item
  \code{cell.reserve(size)}
\item
  \code{cell.reserve(size, item = 0)}
\item
  \code{cell.resize(size, item = 0)}
\item
  \code{result = cell.size()}
\item
  \code{cell.swap(otherCell)}
\end{enumerate}

Iterators are not yet implemented. However, the implementation of the
Python slots \function{__getitem__}, \function{__len__} and
\function{__setitem__} let you use those classes almost as a sequence.
For instance:

\verbatiminput{StdVectorExample.txt}

\section{Function reference \label{functions}}

\Future{}

\begin{cfuncdesc}{const GLubyte*}{gl_error}{}
  is implemented as
  \begin{verbatim}
message = gl_error()
  \end{verbatim}
\end{cfuncdesc}

\begin{cfuncdesc}{bool}{ViewPort2World}
  {double \&wx, double \&wy, double \&wz, double vx, double vy, double vz}
  is implemented as
  \begin{verbatim}
success, wx, wy, wz = ViewPort2World(vx, vy, vz)
  \end{verbatim}
\end{cfuncdesc}

\begin{cfuncdesc}{bool}{World2ViewPort}
  {double \&vx, double \&vy, double \&vz, double wx, double wy, double wz}
  is implemented as
  \begin{verbatim}
success, vx, vy, vz = World2Viewport(wx, wy, wz)
  \end{verbatim}
\end{cfuncdesc}


\documentclass{manual}

% Links
\newcommand{\QwtPlotDdd}{\ulink{QwtPlot3D}
  {http://qwtplot3d.sourceforge.net}}
\newcommand{\QwtPlotDddApi}{\ulink{QwtPlot3D API documentation}
  {http://qwtplot3d.sourceforge.net/web/navigation/api_frame.html}}
\newcommand{\QwtPlotDddManual}{\ulink{QwtPlot3D manual}
  {http://qwtplot3d.sourceforge.net/web/navigation/manual_frame.html}}
\newcommand{\ZLib}{\ulink{ZLib}
  {http://www.gzip.org/zlib}}
\newcommand{\mailinglist}{\ulink{mailing list}
  {mailto:pyqwt-users@lists.sourceforge.net}}

% Source code
\newcommand{\snapshot}{\ulink{snapshot}
  {http://www.river-bank.demon.co.uk/download/snapshots}}
\newcommand{\NumPy}{\ulink{NumPy}
  {http://www.numpy.org}}
\newcommand{\NumPyTarGz}{\ulink{numpy-1.0rc1}
  {http://prdownloads.sourceforge.net/numpy/numpy-1.0rc1.tar.gz?download}}
\newcommand{\numarray}{\ulink{numarray}
  {http://www.stsci.edu/resources/software_hardware/numarray}}
\newcommand{\numarrayTarGz}{\ulink{numarray-1.5.2.tar.gz}
  {http://prdownloads.sourceforge.net/numpy/numarray-1.5.2.tar.gz?download}}
\newcommand{\Numeric}{\ulink{Numeric}
  {http://www.numpy.org}}
\newcommand{\NumericTarGz}{\ulink{Numeric-24.2.tar.gz}
  {http://prdownloads.sourceforge.net/numpy/Numeric-24.2.tar.gz?download}}
\newcommand{\NewSip}{\ulink{sip-4.4.5.tar.gz}
  {http://pyqwt.sourceforge.net/support/sip-4.4.5.tar.gz}}
\newcommand{\SipSnapShot}{\ulink{SIP snapshot}
  {http://www.riverbankcomputing.com/Downloads/Snapshots/sip4/}}
\newcommand{\PyQtGpl}{\ulink{PyQt-x11-gpl-3.16.tar.gz}
  {http://pyqwt.sourceforge.net/support/PyQt-x11-gpl-3.16.tar.gz}}
\newcommand{\PyQtMac}{\ulink{PyQt-mac-gpl-3.16.tar.gz}
  {http://pyqwt.sourceforge.net/support/PyQt-mac-gpl-3.16.tar.gz}}
\newcommand{\PyQtSnapShot}{\ulink{PyQt3 snapshot}
  {http://www.riverbankcomputing.com/Downloads/Snapshots/PyQt3}}
\newcommand{\PyQtFSnapShot}{\ulink{PyQt4 snapshot}
  {http://www.riverbankcomputing.com/Downloads/Snapshots/PyQt4}}
\newcommand{\PyQtFGpl}{\ulink{PyQt4-x11-gpl-4.0.1.tar.gz}
  {http://pyqwt.sourceforge.net/support/PyQt-x11-gpl-4.0.1.tar.gz}}
\newcommand{\PyQtFMac}{\ulink{PyQt4-mac-gpl-4.0.1.tar.gz}
  {http://pyqwt.sourceforge.net/support/PyQt-mac-gpl-4.0.1.tar.gz}}
\newcommand{\PyQtCom}{\ulink{PyQt-commercial}
  {http://www.riverbankcomputing.co.uk/pyqt/buy.php}}
\newcommand{\PyQwtDddTarGz}{\ulink{PyQwt3D-0.1.2.tar.gz}
  {http://prdownloads.sourceforge.net/pyqwt/PyQwt3D-0.1.2.tar.gz?download}}
\newcommand{\QwtPlotDddTgz}{\ulink{qwtplot3d-0.2.6.tgz}
  {http://prdownloads.sourceforge.net/qwtplot3d/qwtplot3d-0.2.6.tgz?download}}
\newcommand{\QwtPlotDddZip}{\ulink{qwtplot3d-0.2.6.zip}
  {http://prdownloads.sourceforge.net/qwtplot3d/qwtplot3d-0.2.6.zip?download}}

% Installers for MS-Windows
\newcommand{\PythonMsi}{\ulink{python-2.4.2/msi}
  {http://python.org/ftp/python/2.4/python-2.4.2.msi}}
\newcommand{\NumericExe}{\ulink{Numeric-24.2.win32-py2.4.exe}
  {http://prdownloads.sourceforge.net/numpy/Numeric-24.2.win32-py2.4.exe?download}}
\newcommand{\numarrayExe}{\ulink{numarray-1.5.2.win32-py2.4.exe}
  {http://prdownloads.sourceforge.net/numpy/numarray-1.5.2.win32-py2.4.exe?download}}
\newcommand{\NumPyExe}{\ulink{numpy-1.0.0.win32-py2.4.exe}
  {http://prdownloads.sourceforge.net/numpy/numpy-1.0.0.win32-py2.4.exe?download}}


\newcommand{\PrerequisitesEnd}{
  To exploit the full power of the PyQwt3D, you should install at
  least one the Numerical Python extensions:
  \NumPy{}, \numarray{}, \Numeric{}.
  I am using \NumPyTarGz{}, \NumericTarGz{} and \numarrayTarGz{}.

  \begin{notice}[note]
    The PyQwt3D extension module containing statically linked sources of
    QwtPlot3D-0.2.6 coexists very well with system wide shared libraries
    of any version of QwtPlot3D.
  \end{notice}

  \begin{notice}[warning]
    PyQwt3D may not work with numarray on Linux systems, possibly due to a bug
    in the floating point excepion handling of glibc. 
    More information is to be found
    \ulink{here}{http://sourceforge.net/mailarchive/message.php?msg_id=9914816}
    and in related posts.
    PyQwt3D does not work with numarray and Matrox graphic cards on
    Mandrake-10.0 and SuSE-9.0.
    Your mileage may vary: PyQwt3D works with numarray and an ATI graphic card
    on SuSE-9.1.
  \end{notice}
}

\newcommand{\Future}{
  \begin{notice}[warning]
    The documentation is for the future PyQwt3D-0.1.2 which is only available
    from CVS.
  \end{notice}
}

%\renewcommand{\Future}{}


\title{PyQwt3D Manual}

% boilerplate.tex?
\author{Gerard Vermeulen}

\date{\today}
\release{0.1.2}
\setshortversion{0.1.2}

\makeindex

\begin{document}

\maketitle

% This makes the contents more accessible from the front page of the HTML.
\ifhtml
\chapter*{Front Matter \label{front}}
\fi

Copyright \copyright{} 2004 Gerard Vermeulen

PyQwt3D is free software; you can redistribute it and/or modify it under the
terms of the GNU General Public License as published by the Free Software
Foundation; either version 2 of the License, or (at your option) any later
version.

PyQwt3D is distributed in the hope that it will be useful, but WITHOUT ANY
WARRANTY; without even the implied warranty of MERCHANTABILITY or FITNESS
FOR A PARTICULAR PURPOSE.  See the GNU  General Public License for more
details.

You should have received a copy of the GNU General Public License along with
PyQwt3D; if not, write to the Free Software Foundation, Inc., 59 Temple Place,
Suite 330, Boston, MA 02111-1307, USA.

In addition, as a special exception, Gerard Vermeulen gives permission to
link PyQwt3D dynamically with commercial, non-commercial or educational 
versions of Qt, PyQt and sip, and distribute PyQwt3D in this form, provided
that equally powerful versions of Qt, PyQt and sip have been released under
the terms of the GNU General Public License.

If PyQwt3D is dynamically linked with commercial, non-commercial or
educational versions of Qt, PyQt and sip, PyQwt3D becomes a free plug-in
for a non-free program.



\begin{abstract}

\noindent
PyQwt3D is a set of Python bindings for the \QwtPlotDdd{} library which extends
the Qt framework with widgets to visualize 3-dimensional data.
It allows you to integrate PyQt, Qt, QwtPlot3D, the Numerical Python
extensions, and optionally SciPy in a GUI Python application or in an
interactive Python session.

\end{abstract}

\tableofcontents

\chapter{Introduction\label{introduction}}

\Future{}

PyQwt3D is a set of Python bindings for the \QwtPlotDdd{} library which extends
the Qt framework with widgets to visualize 3-dimensional data.


\chapter{Installation\label{installation}}

\Future{}

\begin{notice}[note]
  PyQwt3D for Qt-3.3.x can coexist with PyQwt3D for Qt-4.1.x. The statement
  \begin{verbatim}
import Qwt3D
  \end{verbatim}
  imports PyQwt3D for Qt-3.3.x and the statement
  \begin{verbatim}
import PyQt4.Qwt3D
  \end{verbatim}
  imports PyQwt3D for Qt-4.1.x. 
\end{notice}


\section{Installation prerequisites\label{prerequisites}}

\Future{}

Installation prerequisites for \PyQwtDddTarGz{} are:
\begin{enumerate}
\item
  \ulink{Python}{http://www.python.org}.\\
  Supported versions: Python-2.5.x, Python-2.4.x and Python-2.3.x.
\item
  \ulink{Qt}{http://www.trolltech.com}.\\
  Supported versions: Qt-4.2.x, Qt-4.1.x, Qt-3.3.x, or Qt-3.2.x.
\item
  \NewSip{}.\\
  Supported versions: SIP-4.4.x.\\
  Python-2.5 requires a \SipSnapShot{}.\\
\item
  \PyQtGpl{}, \PyQtFGpl, \PyQtMac{}, \PyQtFMac{} or \PyQtCom{}.\\
  Supported versions: PyQt-4.0.x, or PyQt-3.16.x.\\
  Python-2.5 requires a \PyQtSnapShot{} or \PyQtFSnapShot{}.\\
\item
  \QwtPlotDdd{}.\\
  Supported versions: \QwtPlotDddTgz{} or \QwtPlotDddZip{}.
  \PyQwtDddTarGz{} contains a version of \QwtPlotDdd{} for your convenience.
  You can (but unless you are using Windows do not have to) compile and link
  the \QwtPlotDdd{} sources statically into the PyQwt3D extension module.
\item
  \ZLib{} is needed to enable compression in the PDF and PostScript output of
  \QwtPlotDdd{}.\\
  Supported versions: zlib-1.2.x, and zlib-1.1.x.
  \PyQwtDddTarGz{} contains the necessary source files of zlib-1.2.3 to remove
  the dependency on zlib (but you are free to use an already installed shared
  or dynamic load library of zlib).
\end{enumerate}

\PrerequisitesEnd{}


\section{Installation\label{install}}

\Future{}

\subsection{Installation on \POSIX{} and MacOS/X\label{posix-install}}

\Future{}

The installation procedure consists of three steps:
\begin{enumerate}
\item
  Unpack \PyQwtDddTarGz{}.
\item
  Do a quick start to test the installation by running the commands:
\begin{verbatim}
cd PyQwt3D-0.1.2
cd configure
python configure.py -Q ../qwtplot3d-0.2.6
make
make install
\end{verbatim}
  where the directory
  \file{/sources/of/qwtplot3d} must contain the directories \file{3rdparty},
  \file{include} and \file{src}.\\
  The installation will fail if Qt has been configured without runtime type
  information (RTTI).  In this case, run the commands:
\begin{verbatim}
cd PyQwt3D-0.1.2
cd configure
python configure.py -Q /sources/of/qwtplot3d --extra-cxxflags=-frtti
make
make install
\end{verbatim}
  where \code{-frtti} enables RTTI for g++.  Check your compiler documention
  for other C++ compilers.
\item
  Fine tune (optional)
  \begin{itemize}
    \item
      to enable compression of PostScript and PDF files by running the
      commands:
\begin{verbatim}
python configure.py -Q /sources/of/qwtplot3d -l z -D GL2PS_HAVE_ZLIB
make
make install
\end{verbatim}
      Add
\begin{verbatim}
-L /directory/with/libz.*
\end{verbatim}
      to the \file{configure.py} options, if the linker fails to find the zlib
      library.
    \item
      to use a the QwtPlot3D library on your system by running the commands:
\begin{verbatim}
rm -rf Qwt3D
python configure.py -I /usr/include/qwtplot3d
make
make install
\end{verbatim}
      where \file{/usr/include/qwtplot3d} is an example for the installation
      directory of the QwtPlot3D header files.
      Add
\begin{verbatim}
-L /directory/with/libqwtplot3d.*
\end{verbatim}
      to the \file{configure.py} options, if the linker fails to find the
      QwtPlot3D library.
  \end{itemize}
\end{enumerate}

\begin{notice}[note]
  \file{PyQwt-0.1.2/GNUmakefile} is makefile for GNU make which contains more
  examples of how to invoke \file{configure.py}.
  Adapt and use it, if you have GNU make.
\end{notice}

\begin{notice}[note]
  If you run into problems, send a log to the \mailinglist{}.

  There are at least two options to log the output of make:
  \begin{enumerate}
  \item Invoke make, tie stderr to stdout, and redirect stdout to LOG.txt:
\begin{verbatim}
# For Qt-3
make 3 2&>1 >LOG.txt
# For Qt-4
make 4 2&>1 >LOG.txt
\end{verbatim}
    However, you do not see what is going on.
  \item Use script to capture all screen output of make to LOG.txt:
\begin{verbatim}
# For Qt-3
script -c 'make 3' LOG.txt
# For Qt-4
script -c 'make 4' LOG.txt
\end{verbatim}
    The script command appeared in 3.0BSD and is part of util-linux.
  \end{enumerate}
\end{notice}

\begin{notice}[note]
  The configure.py script takes many options. The command
\begin{verbatim}
python configure.py -h
\end{verbatim}
  displays a full list of the available options:
  \verbatiminput{configure.help}
\end{notice}


\subsection{Installation on Windows with MSVC\label{win-install}}

\Future{}

The installation procedure consists of three steps:
\begin{enumerate}
\item
  Unpack \PyQwtDddTarGz{}.
\item
  Do a quick start to test the installation by running the commands:
\begin{verbatim}
cd PyQwt3D-0.1.2
cd configure
python configure.py -Q ..\qwtplot3d-0.2.6
nmake
nmake install
\end{verbatim}
  where the folder
  \file{C:\textbackslash{}sources\textbackslash{}of\textbackslash{}qwtplot3d}
  must contain the folders \file{3rdparty}, \file{include} and \file{src}.
  You can also edit the files \file{go3.bat} or \file{go4.bat} to suit your
  setup.
\item
  Fine tune (optional) by running the commands:
\begin{verbatim}
python configure.py -Q C:\sources\of\qwtplot3d -l zlib -D GL2PS_HAVE_ZLIB
nmake
nmake install
\end{verbatim}
    to enable compression of PostScript and PDF files. Add
\begin{verbatim}
-L C:\folder\containing\zlib.lib
\end{verbatim}
    to the \file{configure.py} options, if the linker fails to find the zlib
    library.
\end{enumerate}

\begin{notice}[note]
  The files \file{configure\textbackslash{}go3.bat} and
  \file{configure\textbackslash{}go4.bat} are examples of how to automatize
  the invokations of \strong{configure.py}, \strong{nmake}, and
  \strong{nmake install}.  Adapt and use it.
\end{notice}

\begin{notice}[note]
  If you run into problems, send a log to the \mailinglist{}.

  Try
\begin{verbatim}
go3.bat >LOG.txt
\end{verbatim}
  or
\begin{verbatim}
go4.bat >LOG.txt
\end{verbatim}
  to make a log.
\end{notice}

\begin{notice}[note]
  The configure.py script takes many options. The command
\begin{verbatim}
python configure.py -h
\end{verbatim}
  displays a full list of the available options:
  \verbatiminput{configure.help}
\end{notice}

\begin{notice}[note]
  Since PyQwt3D wraps some classes and functions that are not exported from
  a QwtPlot3D dynamic load library, you have to compile and link the QwtPlot3D
  sources into PyQwt3D's extension module.
\end{notice}


\chapter{PyQwt3D Module Reference \label{reference}}

\Future{}

The reference should be used in conjunction with the \QwtPlotDddManual{}
and the \QwtPlotDddApi{}.
Only the differences specific to the Python bindings are documented here.

In this chapter, \emph{is not yet implemented} implies that the feature can
be easily implemented if needed, \emph{is not implemented} implies that the
feature is not easily implemented, and \emph{is not Pythonic} implies that
the feature will not be implemented because it violates the Python philosophy
(e.g. may use dangling pointers).

If a class is described as being \emph{fully implemented} then all non-private
member functions and all public class variables have been implemented.

Undocumented classes have not yet been implemented or are still experimental.

The classes in the QwtPlot3D library have quite a few protected attributes.
They are not easily exported to Python (SIP wraps protected member function,
but not protected attributes).
I will export protected attributes to Python on demand. For instance,
\var{Enrichment.plot} is accessible from Python, but protected in C++.




\section{Class reference \label{classes}}

\Future{}

\begin{classdesc*}{Arrow}
  is fully implemented.
\end{classdesc*}

\begin{classdesc*}{AutoScaler}
  is fully implemented.
\end{classdesc*}

\begin{classdesc*}{Axis}
  is fully implemented.
\end{classdesc*}

\begin{classdesc*}{AxisVector}
  wraps \ctype{std::vector<Axis>}. See \ref{wrappers} for details.
\end{classdesc*}

\begin{classdesc*}{Cell}
  wraps \ctype{std::vector<unsigned>}. See \ref{wrappers} for details.
\end{classdesc*}

\begin{classdesc*}{CellData}
  is fully implemented.
\end{classdesc*}

\begin{classdesc*}{CellField}
  wraps \ctype{std::vector<Cell>}. See \ref{wrappers} for details.
\end{classdesc*}

\begin{classdesc*}{Color}
  is fully implemented.
\end{classdesc*}

\begin{classdesc*}{ColorLegend}
  is fully implemented.
\end{classdesc*}

\begin{classdesc*}{Cone}
  is fully implemented.
\end{classdesc*}

\begin{classdesc*}{CoordinateSystem}
  is fully implemented.
\end{classdesc*}

\begin{classdesc*}{CrossHair}
  is fully implemented.
\end{classdesc*}

\begin{classdesc*}{Data}
  is fully implemented.\\
  FIXME: what to do with the protected data members?
\end{classdesc*}

\begin{classdesc*}{Dot}
  is fully implemented.
\end{classdesc*}

\begin{classdesc*}{DoubleVector}
  wraps \ctype{std::vector<double>}. See \ref{wrappers} for details.
\end{classdesc*}

\begin{classdesc*}{Drawable}
  is fully implemented.\\
  FIXME: what to do with the protected data members?
\end{classdesc*}

\begin{classdesc*}{Enrichment}
  is fully implemented.\\
  \begin{cvardesc}{const Plot3D*}{plot}
    This C++ protected data member is accessible in Python.
  \end{cvardesc}
\end{classdesc*}

\begin{classdesc*}{Freevector}
  is fully implemented.
\end{classdesc*}

\begin{classdesc*}{FreeVectorField}
  wraps \ctype{std::vector<FreeVector>}. See \ref{wrappers} for details.
\end{classdesc*}

\begin{classdesc*}{Function}
  is fully implemented.
\end{classdesc*}

\begin{classdesc*}{GLStateBewarer}
  is fully implemented.
\end{classdesc*}

\begin{classdesc*}{GridData}
  \begin{itemize}
  \item{vertices}. The public data member:
\begin{verbatim}
DataMatrix vertices;
\end{verbatim}
    is not accessible. FIXME: how to wrap \class{DataMatrix} safely?
  \item{normals}. The public data member:
\begin{verbatim}
DataMatrix normals;
\end{verbatim}
    is not accessible. FIXME: how to wrap \class{DataMatrix} safely?
  \end{itemize}
\end{classdesc*}

\begin{classdesc*}{GridMapping}
  is fully implemented.\\
  FIXME: what to do with the protected data members?
\end{classdesc*}

\begin{classdesc*}{IO}
  \begin{itemize}
  \item{defineInputHandler}. C++ declaration:
\begin{verbatim}
static bool defineInputHandler(QString const& format, Function func);
\end{verbatim}
    is not implemented (it is impossible to implement callbacks without an
    extra void pointer to hold a Python callable).
  \item{defineOutputHandler}. C++ declaration:
\begin{verbatim}
static bool defineOutputHandler(QString const& format, Function func);
\end{verbatim}
    is not implemented (it is impossible to implement callbacks without an
    extra void pointer to hold a Python callable).
  \end{itemize}
\end{classdesc*}

\begin{classdesc*}{Label}
  is fully implemented.
\end{classdesc*}

\begin{classdesc*}{LinearAutoscaler}
  is fully implemented.
\end{classdesc*}

\begin{classdesc*}{LinearScale}
  is fully implemented.\\
  FIXME: what to do with the protected data members?
\end{classdesc*}

\begin{classdesc*}{LogScale}
  is fully implemented.
\end{classdesc*}

\begin{classdesc*}{Mapping}
  is fully implemented.
\end{classdesc*}

\begin{classdesc*}{NativeReader}
  is fully implemented.
\end{classdesc*}

\begin{classdesc*}{ParallelEpiped}
  is fully implemented.
\end{classdesc*}

\begin{classdesc*}{ParametricSurface}
  is fully implemented.
\end{classdesc*}

\begin{classdesc*}{PixmapWriter}
  is fully implemented.
\end{classdesc*}

\begin{classdesc*}{Plot3D}
  is fully implemented.\\
  FIXME: what to do with the protected data members?\\
\end{classdesc*}

\begin{classdesc*}{RGBA}
  is fully implemented.
\end{classdesc*}

\begin{classdesc*}{Scale}
  is fully implemented.\\
  FIXME: what to do with the protected data members?
\end{classdesc*}

\begin{classdesc*}{StandardColor}
  is fully implemented.\\
  FIXME: what to do with the protected data members?
\end{classdesc*}

\begin{classdesc*}{SurfacePlot}
  is fully implemented.
  \begin{itemize}
  \item{facets}. C++ declaration:
\begin{verbatim}
std::pair<int,int> facets() const;
\end{verbatim}
    returns a tuple of two Python ints.
  \item{loadFromData}. C++ declaration:
\begin{verbatim}
bool loadFromData(Qwt3D::Triple** data,
                  unsigned int columns, unsigned int rows,
                  bool uperiodic = false, bool vperiodic = false);
\end{verbatim}
    is wrapped by:
\begin{verbatim}
success = surfacePlot.loadFromData(data, uperiodic = False, vperiodic = False)
\end{verbatim}
    where \var{success} is \constant{True} or \constant{False}, \var{data}
    is convertable to a Numeric or numarray array of Python floats with a shape
    (N, M, 3), and \var{uperiodic} and \var{vperiodic} are Python bools.\\
    C++ declaration:
\begin{verbatim}
bool loadFromData(double** data, unsigned int columns, unsigned int rows,
                  double minx, double maxx, double miny, double maxy);
\end{verbatim}
    is wrapped by:
\begin{verbatim}
success = surfacePlot.loadFromData(data, minx, maxx, miny, maxy)
\end{verbatim}
    where \var{success} is \constant{True} or \constant{False}, \var{data}
    is convertable to a Numeric or numarray array of Python floats with a shape
    (N, M), and \var{minx}, \var{maxx}, \var{miny} and \var{maxy} are
    convertable to Python floats.\\
    C++ declaration:
\begin{verbatim}
bool loadFromData(Qwt3D::TripleField const& data,
                  Qwt3D::CellField const& poly);
\end{verbatim}
    is wrapped by:
\begin{verbatim}
success = surfacePlot.loadFromData(tripleField, cellField)
\end{verbatim}
    where \var{success} is \constant{True} or \constant{False},
    \var{tripleField} is a \class{TripleField}, and \var{cellField} is a
    \class{CellField}.
  \item{createDataRepresentation}. C++ declarations:
\begin{verbatim}
bool createDataRepresentation(
     Qwt3D::Triple** data, unsigned int columns, unsigned int rows,
     bool uperiodic = false, bool vperiodic = false);
bool createDataRepresentation(
     double** data, unsigned int columns, unsigned int rows,
     double minx, double maxx, double miny, double maxy);
bool createDataRepresentation(
     Qwt3D::TripleField const& data, Qwt3D::CellFieldconst& poly)
\end{verbatim}
    are deprecated and therefore not implemented.
  \item{readIn}. C++ declaration:
\begin{verbatim}
void readIn(Qwt3D::GridData& grid, Triple** data,
            unsigned int columns, unsigned int rows);
\end{verbatim}
    is wrapped by:
\begin{verbatim}
surfacePlot.readIn(gridData, data) 
\end{verbatim}
    where \var{gridData} is a \class{GridData}, and \var{data} is convertable
    to a Numeric or numarray array of Python floats with a shape (N, M, 3).\\
    C++ declaration:
\begin{verbatim}
void readIn(Qwt3D::GridData& grid, double** data,
            unsigned int columns, unsigned int rows,
            double minx, double maxx, double miny, double maxy);
\end{verbatim}
    is wrapped by:
\begin{verbatim}
surfacePlot.readIn(gridData, data, minx, maxx, miny, maxy)
\end{verbatim}
    where \var{gridData} is a \class{GridData}, \var{data} is convertable to
    a Numeric or numarray array of Python floats with a shape (N, M), and
    \var{minx}, \var{maxx}, \var{miny} and \var{maxy} are convertable to Python
    floats.
  \end{itemize}
\end{classdesc*}

\begin{classdesc*}{Triple}
  \begin{itemize}
  \item{operator *()}. C++ declaration:
\begin{verbatim}
Triple operator*(double, const Triple &);
\end{verbatim}
    is not implemented (not supported by SIP-4.2.x).
  \item{operator /()}. C++ declaration:
\begin{verbatim}
Triple operator/(double, const Triple &);
\end{verbatim}
    is not implemented (not supported by SIP-4.2.x).
  \end{itemize}
\end{classdesc*}

\begin{classdesc*}{TripleField}
wraps \ctype{std::vector<Triple>}. See \ref{wrappers} for details.
\end{classdesc*}

\begin{classdesc*}{Tuple}
is fully implemented.
\end{classdesc*}

\begin{classdesc*}{VectorWriter}
is fully implemented.
\end{classdesc*}

\begin{classdesc*}{VertexEnrichment}
is fully implemented.
\end{classdesc*}

\section{Wrappers for \ctype{std::vector<T>} \label{wrappers}}

\Future{}

PyQwt3D has a partial interface to the following C++ std::vector templates:
\begin{enumerate}
\item
  \class{AxisVector} for \ctype{std::vector<Axis>}
\item
  \class{Cell} for \ctype{std::vector<unsigned>}
\item
  \class{CellField} for \ctype{std::vector<Cell>}
\item
  \class{ColorVector} for \ctype{std::vector<RGBA>}
\item
  \class{DoubleVector} for \ctype{std::vector<double>}
\item
  \class{FreeVectorField} for \ctype{std::vector<FreeVectorField>}
\item
  \class{TripleField} for \ctype{std::vector<Triple>}
\end{enumerate}

The interface implements four constructors for each template instantianation --
taking Cell as example:
\begin{enumerate}
\item
  \code{Cell()}
\item
  \code{Cell(size)}
\item
  \code{Cell(size, item)}
\item
  \code{Cell(otherCell)}
\end{enumerate}

and 13 member functions -- taking Cell as example:
\begin{enumerate}
\item
  \code{result = cell.capacity()}
\item
  \code{cell.clear()}
\item
  \code{result = cell.empty()}
\item
  \code{result = cell.back()}
\item
  \code{result = cell.front()}
\item
  \code{result = cell.max_size()}
\item
  \code{cell.pop_back()}
\item
  \code{cell.push_back(item)}
\item
  \code{cell.reserve(size)}
\item
  \code{cell.reserve(size, item = 0)}
\item
  \code{cell.resize(size, item = 0)}
\item
  \code{result = cell.size()}
\item
  \code{cell.swap(otherCell)}
\end{enumerate}

Iterators are not yet implemented. However, the implementation of the
Python slots \function{__getitem__}, \function{__len__} and
\function{__setitem__} let you use those classes almost as a sequence.
For instance:

\verbatiminput{StdVectorExample.txt}

\section{Function reference \label{functions}}

\Future{}

\begin{cfuncdesc}{const GLubyte*}{gl_error}{}
  is implemented as
  \begin{verbatim}
message = gl_error()
  \end{verbatim}
\end{cfuncdesc}

\begin{cfuncdesc}{bool}{ViewPort2World}
  {double \&wx, double \&wy, double \&wz, double vx, double vy, double vz}
  is implemented as
  \begin{verbatim}
success, wx, wy, wz = ViewPort2World(vx, vy, vz)
  \end{verbatim}
\end{cfuncdesc}

\begin{cfuncdesc}{bool}{World2ViewPort}
  {double \&vx, double \&vy, double \&vz, double wx, double wy, double wz}
  is implemented as
  \begin{verbatim}
success, vx, vy, vz = World2Viewport(wx, wy, wz)
  \end{verbatim}
\end{cfuncdesc}


\documentclass{manual}

% Links
\newcommand{\QwtPlotDdd}{\ulink{QwtPlot3D}
  {http://qwtplot3d.sourceforge.net}}
\newcommand{\QwtPlotDddApi}{\ulink{QwtPlot3D API documentation}
  {http://qwtplot3d.sourceforge.net/web/navigation/api_frame.html}}
\newcommand{\QwtPlotDddManual}{\ulink{QwtPlot3D manual}
  {http://qwtplot3d.sourceforge.net/web/navigation/manual_frame.html}}
\newcommand{\ZLib}{\ulink{ZLib}
  {http://www.gzip.org/zlib}}
\newcommand{\mailinglist}{\ulink{mailing list}
  {mailto:pyqwt-users@lists.sourceforge.net}}

% Source code
\newcommand{\snapshot}{\ulink{snapshot}
  {http://www.river-bank.demon.co.uk/download/snapshots}}
\newcommand{\NumPy}{\ulink{NumPy}
  {http://www.numpy.org}}
\newcommand{\NumPyTarGz}{\ulink{numpy-1.0rc1}
  {http://prdownloads.sourceforge.net/numpy/numpy-1.0rc1.tar.gz?download}}
\newcommand{\numarray}{\ulink{numarray}
  {http://www.stsci.edu/resources/software_hardware/numarray}}
\newcommand{\numarrayTarGz}{\ulink{numarray-1.5.2.tar.gz}
  {http://prdownloads.sourceforge.net/numpy/numarray-1.5.2.tar.gz?download}}
\newcommand{\Numeric}{\ulink{Numeric}
  {http://www.numpy.org}}
\newcommand{\NumericTarGz}{\ulink{Numeric-24.2.tar.gz}
  {http://prdownloads.sourceforge.net/numpy/Numeric-24.2.tar.gz?download}}
\newcommand{\NewSip}{\ulink{sip-4.4.5.tar.gz}
  {http://pyqwt.sourceforge.net/support/sip-4.4.5.tar.gz}}
\newcommand{\SipSnapShot}{\ulink{SIP snapshot}
  {http://www.riverbankcomputing.com/Downloads/Snapshots/sip4/}}
\newcommand{\PyQtGpl}{\ulink{PyQt-x11-gpl-3.16.tar.gz}
  {http://pyqwt.sourceforge.net/support/PyQt-x11-gpl-3.16.tar.gz}}
\newcommand{\PyQtMac}{\ulink{PyQt-mac-gpl-3.16.tar.gz}
  {http://pyqwt.sourceforge.net/support/PyQt-mac-gpl-3.16.tar.gz}}
\newcommand{\PyQtSnapShot}{\ulink{PyQt3 snapshot}
  {http://www.riverbankcomputing.com/Downloads/Snapshots/PyQt3}}
\newcommand{\PyQtFSnapShot}{\ulink{PyQt4 snapshot}
  {http://www.riverbankcomputing.com/Downloads/Snapshots/PyQt4}}
\newcommand{\PyQtFGpl}{\ulink{PyQt4-x11-gpl-4.0.1.tar.gz}
  {http://pyqwt.sourceforge.net/support/PyQt-x11-gpl-4.0.1.tar.gz}}
\newcommand{\PyQtFMac}{\ulink{PyQt4-mac-gpl-4.0.1.tar.gz}
  {http://pyqwt.sourceforge.net/support/PyQt-mac-gpl-4.0.1.tar.gz}}
\newcommand{\PyQtCom}{\ulink{PyQt-commercial}
  {http://www.riverbankcomputing.co.uk/pyqt/buy.php}}
\newcommand{\PyQwtDddTarGz}{\ulink{PyQwt3D-0.1.2.tar.gz}
  {http://prdownloads.sourceforge.net/pyqwt/PyQwt3D-0.1.2.tar.gz?download}}
\newcommand{\QwtPlotDddTgz}{\ulink{qwtplot3d-0.2.6.tgz}
  {http://prdownloads.sourceforge.net/qwtplot3d/qwtplot3d-0.2.6.tgz?download}}
\newcommand{\QwtPlotDddZip}{\ulink{qwtplot3d-0.2.6.zip}
  {http://prdownloads.sourceforge.net/qwtplot3d/qwtplot3d-0.2.6.zip?download}}

% Installers for MS-Windows
\newcommand{\PythonMsi}{\ulink{python-2.4.2/msi}
  {http://python.org/ftp/python/2.4/python-2.4.2.msi}}
\newcommand{\NumericExe}{\ulink{Numeric-24.2.win32-py2.4.exe}
  {http://prdownloads.sourceforge.net/numpy/Numeric-24.2.win32-py2.4.exe?download}}
\newcommand{\numarrayExe}{\ulink{numarray-1.5.2.win32-py2.4.exe}
  {http://prdownloads.sourceforge.net/numpy/numarray-1.5.2.win32-py2.4.exe?download}}
\newcommand{\NumPyExe}{\ulink{numpy-1.0.0.win32-py2.4.exe}
  {http://prdownloads.sourceforge.net/numpy/numpy-1.0.0.win32-py2.4.exe?download}}


\newcommand{\PrerequisitesEnd}{
  To exploit the full power of the PyQwt3D, you should install at
  least one the Numerical Python extensions:
  \NumPy{}, \numarray{}, \Numeric{}.
  I am using \NumPyTarGz{}, \NumericTarGz{} and \numarrayTarGz{}.

  \begin{notice}[note]
    The PyQwt3D extension module containing statically linked sources of
    QwtPlot3D-0.2.6 coexists very well with system wide shared libraries
    of any version of QwtPlot3D.
  \end{notice}

  \begin{notice}[warning]
    PyQwt3D may not work with numarray on Linux systems, possibly due to a bug
    in the floating point excepion handling of glibc. 
    More information is to be found
    \ulink{here}{http://sourceforge.net/mailarchive/message.php?msg_id=9914816}
    and in related posts.
    PyQwt3D does not work with numarray and Matrox graphic cards on
    Mandrake-10.0 and SuSE-9.0.
    Your mileage may vary: PyQwt3D works with numarray and an ATI graphic card
    on SuSE-9.1.
  \end{notice}
}

\newcommand{\Future}{
  \begin{notice}[warning]
    The documentation is for the future PyQwt3D-0.1.2 which is only available
    from CVS.
  \end{notice}
}

%\renewcommand{\Future}{}


\title{PyQwt3D Manual}

% boilerplate.tex?
\author{Gerard Vermeulen}

\date{\today}
\release{0.1.2}
\setshortversion{0.1.2}

\makeindex

\begin{document}

\maketitle

% This makes the contents more accessible from the front page of the HTML.
\ifhtml
\chapter*{Front Matter \label{front}}
\fi

\input{copyright}


\begin{abstract}

\noindent
PyQwt3D is a set of Python bindings for the \QwtPlotDdd{} library which extends
the Qt framework with widgets to visualize 3-dimensional data.
It allows you to integrate PyQt, Qt, QwtPlot3D, the Numerical Python
extensions, and optionally SciPy in a GUI Python application or in an
interactive Python session.

\end{abstract}

\tableofcontents

\chapter{Introduction\label{introduction}}

\Future{}

PyQwt3D is a set of Python bindings for the \QwtPlotDdd{} library which extends
the Qt framework with widgets to visualize 3-dimensional data.


\chapter{Installation\label{installation}}

\Future{}

\begin{notice}[note]
  PyQwt3D for Qt-3.3.x can coexist with PyQwt3D for Qt-4.1.x. The statement
  \begin{verbatim}
import Qwt3D
  \end{verbatim}
  imports PyQwt3D for Qt-3.3.x and the statement
  \begin{verbatim}
import PyQt4.Qwt3D
  \end{verbatim}
  imports PyQwt3D for Qt-4.1.x. 
\end{notice}


\section{Installation prerequisites\label{prerequisites}}

\Future{}

Installation prerequisites for \PyQwtDddTarGz{} are:
\begin{enumerate}
\item
  \ulink{Python}{http://www.python.org}.\\
  Supported versions: Python-2.5.x, Python-2.4.x and Python-2.3.x.
\item
  \ulink{Qt}{http://www.trolltech.com}.\\
  Supported versions: Qt-4.2.x, Qt-4.1.x, Qt-3.3.x, or Qt-3.2.x.
\item
  \NewSip{}.\\
  Supported versions: SIP-4.4.x.\\
  Python-2.5 requires a \SipSnapShot{}.\\
\item
  \PyQtGpl{}, \PyQtFGpl, \PyQtMac{}, \PyQtFMac{} or \PyQtCom{}.\\
  Supported versions: PyQt-4.0.x, or PyQt-3.16.x.\\
  Python-2.5 requires a \PyQtSnapShot{} or \PyQtFSnapShot{}.\\
\item
  \QwtPlotDdd{}.\\
  Supported versions: \QwtPlotDddTgz{} or \QwtPlotDddZip{}.
  \PyQwtDddTarGz{} contains a version of \QwtPlotDdd{} for your convenience.
  You can (but unless you are using Windows do not have to) compile and link
  the \QwtPlotDdd{} sources statically into the PyQwt3D extension module.
\item
  \ZLib{} is needed to enable compression in the PDF and PostScript output of
  \QwtPlotDdd{}.\\
  Supported versions: zlib-1.2.x, and zlib-1.1.x.
  \PyQwtDddTarGz{} contains the necessary source files of zlib-1.2.3 to remove
  the dependency on zlib (but you are free to use an already installed shared
  or dynamic load library of zlib).
\end{enumerate}

\PrerequisitesEnd{}


\section{Installation\label{install}}

\Future{}

\subsection{Installation on \POSIX{} and MacOS/X\label{posix-install}}

\Future{}

The installation procedure consists of three steps:
\begin{enumerate}
\item
  Unpack \PyQwtDddTarGz{}.
\item
  Do a quick start to test the installation by running the commands:
\begin{verbatim}
cd PyQwt3D-0.1.2
cd configure
python configure.py -Q ../qwtplot3d-0.2.6
make
make install
\end{verbatim}
  where the directory
  \file{/sources/of/qwtplot3d} must contain the directories \file{3rdparty},
  \file{include} and \file{src}.\\
  The installation will fail if Qt has been configured without runtime type
  information (RTTI).  In this case, run the commands:
\begin{verbatim}
cd PyQwt3D-0.1.2
cd configure
python configure.py -Q /sources/of/qwtplot3d --extra-cxxflags=-frtti
make
make install
\end{verbatim}
  where \code{-frtti} enables RTTI for g++.  Check your compiler documention
  for other C++ compilers.
\item
  Fine tune (optional)
  \begin{itemize}
    \item
      to enable compression of PostScript and PDF files by running the
      commands:
\begin{verbatim}
python configure.py -Q /sources/of/qwtplot3d -l z -D GL2PS_HAVE_ZLIB
make
make install
\end{verbatim}
      Add
\begin{verbatim}
-L /directory/with/libz.*
\end{verbatim}
      to the \file{configure.py} options, if the linker fails to find the zlib
      library.
    \item
      to use a the QwtPlot3D library on your system by running the commands:
\begin{verbatim}
rm -rf Qwt3D
python configure.py -I /usr/include/qwtplot3d
make
make install
\end{verbatim}
      where \file{/usr/include/qwtplot3d} is an example for the installation
      directory of the QwtPlot3D header files.
      Add
\begin{verbatim}
-L /directory/with/libqwtplot3d.*
\end{verbatim}
      to the \file{configure.py} options, if the linker fails to find the
      QwtPlot3D library.
  \end{itemize}
\end{enumerate}

\begin{notice}[note]
  \file{PyQwt-0.1.2/GNUmakefile} is makefile for GNU make which contains more
  examples of how to invoke \file{configure.py}.
  Adapt and use it, if you have GNU make.
\end{notice}

\begin{notice}[note]
  If you run into problems, send a log to the \mailinglist{}.

  There are at least two options to log the output of make:
  \begin{enumerate}
  \item Invoke make, tie stderr to stdout, and redirect stdout to LOG.txt:
\begin{verbatim}
# For Qt-3
make 3 2&>1 >LOG.txt
# For Qt-4
make 4 2&>1 >LOG.txt
\end{verbatim}
    However, you do not see what is going on.
  \item Use script to capture all screen output of make to LOG.txt:
\begin{verbatim}
# For Qt-3
script -c 'make 3' LOG.txt
# For Qt-4
script -c 'make 4' LOG.txt
\end{verbatim}
    The script command appeared in 3.0BSD and is part of util-linux.
  \end{enumerate}
\end{notice}

\begin{notice}[note]
  The configure.py script takes many options. The command
\begin{verbatim}
python configure.py -h
\end{verbatim}
  displays a full list of the available options:
  \verbatiminput{configure.help}
\end{notice}


\subsection{Installation on Windows with MSVC\label{win-install}}

\Future{}

The installation procedure consists of three steps:
\begin{enumerate}
\item
  Unpack \PyQwtDddTarGz{}.
\item
  Do a quick start to test the installation by running the commands:
\begin{verbatim}
cd PyQwt3D-0.1.2
cd configure
python configure.py -Q ..\qwtplot3d-0.2.6
nmake
nmake install
\end{verbatim}
  where the folder
  \file{C:\textbackslash{}sources\textbackslash{}of\textbackslash{}qwtplot3d}
  must contain the folders \file{3rdparty}, \file{include} and \file{src}.
  You can also edit the files \file{go3.bat} or \file{go4.bat} to suit your
  setup.
\item
  Fine tune (optional) by running the commands:
\begin{verbatim}
python configure.py -Q C:\sources\of\qwtplot3d -l zlib -D GL2PS_HAVE_ZLIB
nmake
nmake install
\end{verbatim}
    to enable compression of PostScript and PDF files. Add
\begin{verbatim}
-L C:\folder\containing\zlib.lib
\end{verbatim}
    to the \file{configure.py} options, if the linker fails to find the zlib
    library.
\end{enumerate}

\begin{notice}[note]
  The files \file{configure\textbackslash{}go3.bat} and
  \file{configure\textbackslash{}go4.bat} are examples of how to automatize
  the invokations of \strong{configure.py}, \strong{nmake}, and
  \strong{nmake install}.  Adapt and use it.
\end{notice}

\begin{notice}[note]
  If you run into problems, send a log to the \mailinglist{}.

  Try
\begin{verbatim}
go3.bat >LOG.txt
\end{verbatim}
  or
\begin{verbatim}
go4.bat >LOG.txt
\end{verbatim}
  to make a log.
\end{notice}

\begin{notice}[note]
  The configure.py script takes many options. The command
\begin{verbatim}
python configure.py -h
\end{verbatim}
  displays a full list of the available options:
  \verbatiminput{configure.help}
\end{notice}

\begin{notice}[note]
  Since PyQwt3D wraps some classes and functions that are not exported from
  a QwtPlot3D dynamic load library, you have to compile and link the QwtPlot3D
  sources into PyQwt3D's extension module.
\end{notice}


\chapter{PyQwt3D Module Reference \label{reference}}

\Future{}

The reference should be used in conjunction with the \QwtPlotDddManual{}
and the \QwtPlotDddApi{}.
Only the differences specific to the Python bindings are documented here.

In this chapter, \emph{is not yet implemented} implies that the feature can
be easily implemented if needed, \emph{is not implemented} implies that the
feature is not easily implemented, and \emph{is not Pythonic} implies that
the feature will not be implemented because it violates the Python philosophy
(e.g. may use dangling pointers).

If a class is described as being \emph{fully implemented} then all non-private
member functions and all public class variables have been implemented.

Undocumented classes have not yet been implemented or are still experimental.

The classes in the QwtPlot3D library have quite a few protected attributes.
They are not easily exported to Python (SIP wraps protected member function,
but not protected attributes).
I will export protected attributes to Python on demand. For instance,
\var{Enrichment.plot} is accessible from Python, but protected in C++.




\section{Class reference \label{classes}}

\Future{}

\begin{classdesc*}{Arrow}
  is fully implemented.
\end{classdesc*}

\begin{classdesc*}{AutoScaler}
  is fully implemented.
\end{classdesc*}

\begin{classdesc*}{Axis}
  is fully implemented.
\end{classdesc*}

\begin{classdesc*}{AxisVector}
  wraps \ctype{std::vector<Axis>}. See \ref{wrappers} for details.
\end{classdesc*}

\begin{classdesc*}{Cell}
  wraps \ctype{std::vector<unsigned>}. See \ref{wrappers} for details.
\end{classdesc*}

\begin{classdesc*}{CellData}
  is fully implemented.
\end{classdesc*}

\begin{classdesc*}{CellField}
  wraps \ctype{std::vector<Cell>}. See \ref{wrappers} for details.
\end{classdesc*}

\begin{classdesc*}{Color}
  is fully implemented.
\end{classdesc*}

\begin{classdesc*}{ColorLegend}
  is fully implemented.
\end{classdesc*}

\begin{classdesc*}{Cone}
  is fully implemented.
\end{classdesc*}

\begin{classdesc*}{CoordinateSystem}
  is fully implemented.
\end{classdesc*}

\begin{classdesc*}{CrossHair}
  is fully implemented.
\end{classdesc*}

\begin{classdesc*}{Data}
  is fully implemented.\\
  FIXME: what to do with the protected data members?
\end{classdesc*}

\begin{classdesc*}{Dot}
  is fully implemented.
\end{classdesc*}

\begin{classdesc*}{DoubleVector}
  wraps \ctype{std::vector<double>}. See \ref{wrappers} for details.
\end{classdesc*}

\begin{classdesc*}{Drawable}
  is fully implemented.\\
  FIXME: what to do with the protected data members?
\end{classdesc*}

\begin{classdesc*}{Enrichment}
  is fully implemented.\\
  \begin{cvardesc}{const Plot3D*}{plot}
    This C++ protected data member is accessible in Python.
  \end{cvardesc}
\end{classdesc*}

\begin{classdesc*}{Freevector}
  is fully implemented.
\end{classdesc*}

\begin{classdesc*}{FreeVectorField}
  wraps \ctype{std::vector<FreeVector>}. See \ref{wrappers} for details.
\end{classdesc*}

\begin{classdesc*}{Function}
  is fully implemented.
\end{classdesc*}

\begin{classdesc*}{GLStateBewarer}
  is fully implemented.
\end{classdesc*}

\begin{classdesc*}{GridData}
  \begin{itemize}
  \item{vertices}. The public data member:
\begin{verbatim}
DataMatrix vertices;
\end{verbatim}
    is not accessible. FIXME: how to wrap \class{DataMatrix} safely?
  \item{normals}. The public data member:
\begin{verbatim}
DataMatrix normals;
\end{verbatim}
    is not accessible. FIXME: how to wrap \class{DataMatrix} safely?
  \end{itemize}
\end{classdesc*}

\begin{classdesc*}{GridMapping}
  is fully implemented.\\
  FIXME: what to do with the protected data members?
\end{classdesc*}

\begin{classdesc*}{IO}
  \begin{itemize}
  \item{defineInputHandler}. C++ declaration:
\begin{verbatim}
static bool defineInputHandler(QString const& format, Function func);
\end{verbatim}
    is not implemented (it is impossible to implement callbacks without an
    extra void pointer to hold a Python callable).
  \item{defineOutputHandler}. C++ declaration:
\begin{verbatim}
static bool defineOutputHandler(QString const& format, Function func);
\end{verbatim}
    is not implemented (it is impossible to implement callbacks without an
    extra void pointer to hold a Python callable).
  \end{itemize}
\end{classdesc*}

\begin{classdesc*}{Label}
  is fully implemented.
\end{classdesc*}

\begin{classdesc*}{LinearAutoscaler}
  is fully implemented.
\end{classdesc*}

\begin{classdesc*}{LinearScale}
  is fully implemented.\\
  FIXME: what to do with the protected data members?
\end{classdesc*}

\begin{classdesc*}{LogScale}
  is fully implemented.
\end{classdesc*}

\begin{classdesc*}{Mapping}
  is fully implemented.
\end{classdesc*}

\begin{classdesc*}{NativeReader}
  is fully implemented.
\end{classdesc*}

\begin{classdesc*}{ParallelEpiped}
  is fully implemented.
\end{classdesc*}

\begin{classdesc*}{ParametricSurface}
  is fully implemented.
\end{classdesc*}

\begin{classdesc*}{PixmapWriter}
  is fully implemented.
\end{classdesc*}

\begin{classdesc*}{Plot3D}
  is fully implemented.\\
  FIXME: what to do with the protected data members?\\
\end{classdesc*}

\begin{classdesc*}{RGBA}
  is fully implemented.
\end{classdesc*}

\begin{classdesc*}{Scale}
  is fully implemented.\\
  FIXME: what to do with the protected data members?
\end{classdesc*}

\begin{classdesc*}{StandardColor}
  is fully implemented.\\
  FIXME: what to do with the protected data members?
\end{classdesc*}

\begin{classdesc*}{SurfacePlot}
  is fully implemented.
  \begin{itemize}
  \item{facets}. C++ declaration:
\begin{verbatim}
std::pair<int,int> facets() const;
\end{verbatim}
    returns a tuple of two Python ints.
  \item{loadFromData}. C++ declaration:
\begin{verbatim}
bool loadFromData(Qwt3D::Triple** data,
                  unsigned int columns, unsigned int rows,
                  bool uperiodic = false, bool vperiodic = false);
\end{verbatim}
    is wrapped by:
\begin{verbatim}
success = surfacePlot.loadFromData(data, uperiodic = False, vperiodic = False)
\end{verbatim}
    where \var{success} is \constant{True} or \constant{False}, \var{data}
    is convertable to a Numeric or numarray array of Python floats with a shape
    (N, M, 3), and \var{uperiodic} and \var{vperiodic} are Python bools.\\
    C++ declaration:
\begin{verbatim}
bool loadFromData(double** data, unsigned int columns, unsigned int rows,
                  double minx, double maxx, double miny, double maxy);
\end{verbatim}
    is wrapped by:
\begin{verbatim}
success = surfacePlot.loadFromData(data, minx, maxx, miny, maxy)
\end{verbatim}
    where \var{success} is \constant{True} or \constant{False}, \var{data}
    is convertable to a Numeric or numarray array of Python floats with a shape
    (N, M), and \var{minx}, \var{maxx}, \var{miny} and \var{maxy} are
    convertable to Python floats.\\
    C++ declaration:
\begin{verbatim}
bool loadFromData(Qwt3D::TripleField const& data,
                  Qwt3D::CellField const& poly);
\end{verbatim}
    is wrapped by:
\begin{verbatim}
success = surfacePlot.loadFromData(tripleField, cellField)
\end{verbatim}
    where \var{success} is \constant{True} or \constant{False},
    \var{tripleField} is a \class{TripleField}, and \var{cellField} is a
    \class{CellField}.
  \item{createDataRepresentation}. C++ declarations:
\begin{verbatim}
bool createDataRepresentation(
     Qwt3D::Triple** data, unsigned int columns, unsigned int rows,
     bool uperiodic = false, bool vperiodic = false);
bool createDataRepresentation(
     double** data, unsigned int columns, unsigned int rows,
     double minx, double maxx, double miny, double maxy);
bool createDataRepresentation(
     Qwt3D::TripleField const& data, Qwt3D::CellFieldconst& poly)
\end{verbatim}
    are deprecated and therefore not implemented.
  \item{readIn}. C++ declaration:
\begin{verbatim}
void readIn(Qwt3D::GridData& grid, Triple** data,
            unsigned int columns, unsigned int rows);
\end{verbatim}
    is wrapped by:
\begin{verbatim}
surfacePlot.readIn(gridData, data) 
\end{verbatim}
    where \var{gridData} is a \class{GridData}, and \var{data} is convertable
    to a Numeric or numarray array of Python floats with a shape (N, M, 3).\\
    C++ declaration:
\begin{verbatim}
void readIn(Qwt3D::GridData& grid, double** data,
            unsigned int columns, unsigned int rows,
            double minx, double maxx, double miny, double maxy);
\end{verbatim}
    is wrapped by:
\begin{verbatim}
surfacePlot.readIn(gridData, data, minx, maxx, miny, maxy)
\end{verbatim}
    where \var{gridData} is a \class{GridData}, \var{data} is convertable to
    a Numeric or numarray array of Python floats with a shape (N, M), and
    \var{minx}, \var{maxx}, \var{miny} and \var{maxy} are convertable to Python
    floats.
  \end{itemize}
\end{classdesc*}

\begin{classdesc*}{Triple}
  \begin{itemize}
  \item{operator *()}. C++ declaration:
\begin{verbatim}
Triple operator*(double, const Triple &);
\end{verbatim}
    is not implemented (not supported by SIP-4.2.x).
  \item{operator /()}. C++ declaration:
\begin{verbatim}
Triple operator/(double, const Triple &);
\end{verbatim}
    is not implemented (not supported by SIP-4.2.x).
  \end{itemize}
\end{classdesc*}

\begin{classdesc*}{TripleField}
wraps \ctype{std::vector<Triple>}. See \ref{wrappers} for details.
\end{classdesc*}

\begin{classdesc*}{Tuple}
is fully implemented.
\end{classdesc*}

\begin{classdesc*}{VectorWriter}
is fully implemented.
\end{classdesc*}

\begin{classdesc*}{VertexEnrichment}
is fully implemented.
\end{classdesc*}

\section{Wrappers for \ctype{std::vector<T>} \label{wrappers}}

\Future{}

PyQwt3D has a partial interface to the following C++ std::vector templates:
\begin{enumerate}
\item
  \class{AxisVector} for \ctype{std::vector<Axis>}
\item
  \class{Cell} for \ctype{std::vector<unsigned>}
\item
  \class{CellField} for \ctype{std::vector<Cell>}
\item
  \class{ColorVector} for \ctype{std::vector<RGBA>}
\item
  \class{DoubleVector} for \ctype{std::vector<double>}
\item
  \class{FreeVectorField} for \ctype{std::vector<FreeVectorField>}
\item
  \class{TripleField} for \ctype{std::vector<Triple>}
\end{enumerate}

The interface implements four constructors for each template instantianation --
taking Cell as example:
\begin{enumerate}
\item
  \code{Cell()}
\item
  \code{Cell(size)}
\item
  \code{Cell(size, item)}
\item
  \code{Cell(otherCell)}
\end{enumerate}

and 13 member functions -- taking Cell as example:
\begin{enumerate}
\item
  \code{result = cell.capacity()}
\item
  \code{cell.clear()}
\item
  \code{result = cell.empty()}
\item
  \code{result = cell.back()}
\item
  \code{result = cell.front()}
\item
  \code{result = cell.max_size()}
\item
  \code{cell.pop_back()}
\item
  \code{cell.push_back(item)}
\item
  \code{cell.reserve(size)}
\item
  \code{cell.reserve(size, item = 0)}
\item
  \code{cell.resize(size, item = 0)}
\item
  \code{result = cell.size()}
\item
  \code{cell.swap(otherCell)}
\end{enumerate}

Iterators are not yet implemented. However, the implementation of the
Python slots \function{__getitem__}, \function{__len__} and
\function{__setitem__} let you use those classes almost as a sequence.
For instance:

\verbatiminput{StdVectorExample.txt}

\section{Function reference \label{functions}}

\Future{}

\begin{cfuncdesc}{const GLubyte*}{gl_error}{}
  is implemented as
  \begin{verbatim}
message = gl_error()
  \end{verbatim}
\end{cfuncdesc}

\begin{cfuncdesc}{bool}{ViewPort2World}
  {double \&wx, double \&wy, double \&wz, double vx, double vy, double vz}
  is implemented as
  \begin{verbatim}
success, wx, wy, wz = ViewPort2World(vx, vy, vz)
  \end{verbatim}
\end{cfuncdesc}

\begin{cfuncdesc}{bool}{World2ViewPort}
  {double \&vx, double \&vy, double \&vz, double wx, double wy, double wz}
  is implemented as
  \begin{verbatim}
success, vx, vy, vz = World2Viewport(wx, wy, wz)
  \end{verbatim}
\end{cfuncdesc}


\input{pyqwt3d.ind}

\end{document}

%% Local Variables:
%% fill-column: 79
%% End:


\end{document}

%% Local Variables:
%% fill-column: 79
%% End:


\end{document}

%% Local Variables:
%% fill-column: 79
%% End:


\end{document}

%% Local Variables:
%% fill-column: 79
%% End:
